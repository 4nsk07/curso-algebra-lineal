\documentclass[12p,spanish]{article}
\usepackage[spanish]{babel}
\usepackage[ansinew]{inputenc}
\usepackage[T1]{fontenc}
\usepackage{graphicx}
\usepackage{multicol}
\usepackage{longtable}
\usepackage{array}
\usepackage{multirow}
\usepackage{geometry}                		
\geometry{letterpaper}                   		
\usepackage{graphicx}
\usepackage{amssymb}
\usepackage{color}


\setlength{\textwidth}{16cm}
\setlength{\textheight}{24cm}
\setlength{\oddsidemargin}{-0.3cm}
\setlength{\topmargin}{-1.3cm}


\newcommand{\sC}{{\cal C}}
\newcommand{\sF}{{\cal F}}
\newcommand{\sL}{{\cal L}}
\newcommand{\sU}{{\cal U}}
\newcommand{\sX}{{\cal X}}
\newcommand{\eop}{{\Box}}

\newcommand{\ar}{A^{(r)}}
\newcommand{\HH}{{\bf H}}
\newcommand{\sS}{{\cal S}}
\newcommand{\Img}{\mbox{Img}}

\def\N{I\!\!N}
\def\R{I\!\!R}
\def\Z{Z\!\!\!Z}
\def\Q{O\!\!\!\!Q}
\def\C{C\!\!\!\!I}


\newcount\problemes
\problemes=0

\def\probl{\advance\problemes by 1
\vskip 2ex\noindent{\bf \the\problemes \hbox{ } }}

\graphicspath{ {im/} }


\newcommand{\notimplies}{%
  \mathrel{{\ooalign{\hidewidth$\not\phantom{=}$\hidewidth\cr$\implies$}}}}





\begin{document}
\pagestyle{empty}

\parindent =0 pt
{\bf Problemes T. Aut�nom d'Algebra Lineal. Primer de Telem�tica. 
\hfill Tema 5 - Diagonalitzaci�}

\vspace{0.6 cm}
\probl Calculau  els valors propis de les matrius, aix� com les seves multiplicitats algebraiques i geom�triques dels seus autovalors
\[A = \left(\begin{array}{cc}0 & -1  \\1 & 0\end{array}\right)\]
\[B = \left(\begin{array}{ccc}2 & -4 & -4  \\0 & -2 & 0 \\ -4 & 4 & 2\end{array}\right)\]
\[C = \left(\begin{array}{ccc}-5 & 3 & 3  \\-2 & 4 & 2 \\ -7 & 3 & 5\end{array}\right)\]
\[D = \left(\begin{array}{cccc}1 & 1 &1 &1 \\1 & 1 & -1 & -1\\ 1 & -1 & 1 & -1\\ 1 & -1 & -1 & 1\end{array}\right)\]

\vspace{0.4 cm}
\probl Estudiau la diagonalitzaci� de les matrius quan sigui possible
\[A = \left(\begin{array}{cc}1 & 4  \\0 & 3\end{array}\right)\]
\[B = \left(\begin{array}{ccc}-1 & -7 & 1  \\0 & 4 & 0 \\ -1 & 13 & -3\end{array}\right)\]
\[C = \left(\begin{array}{ccc}1 & -3 & 3  \\3 & -5 & 3 \\ 6 & -6 & 4\end{array}\right)\]

\vspace{0.4 cm}
\probl Sigui la matriu
\[A = \left(\begin{array}{ccc}1 & -1 & -1  \\-1 & 1 & -1 \\ -1 & -1 & 1\end{array}\right)\]
Trobau els valors i els vectors propis. Determinau els subespais propis associats. Diagonalitzau la matriu $A$ si es possible. 

\vspace{0.4 cm}
\probl Sigui la matriu
\[A = \left(\begin{array}{ccc}0 & 0 & 4  \\1 & 2 & 1 \\ 2 & 4 & -2\end{array}\right)\]
Trobau els valors i els vectors propis. Determinau els subespais propis associats. Diagonalitzau la matriu $A$ si es possible. 

\vspace{0.4 cm}
\probl Sigui la matriu
\[A = \left(\begin{array}{ccc}3 & -1 & 1  \\0 & 2 & 0 \\ 1 & -1 & 3\end{array}\right)\]
Trobau una base de $\mathbb R^3$ formada per vectors propis de $A$.

\vspace{0.4 cm}
\probl Estudiau els valors dels par�metres pels quals s�n diagonalitzables les seg�ents matrius:
\[A = \left(\begin{array}{ccc}1 & -4 & 0  \\0 & 4a & 0 \\ 0 & 0 & 3\end{array}\right)\]
\[B = \left(\begin{array}{ccc}a & 2 & 0  \\0 & -1 & 0 \\ 0 & 0 & 1\end{array}\right)\]
\[C = \left(\begin{array}{ccc}3 & 0 & -1  \\-1 & 3 & 0 \\ 0 & 0 & t\end{array}\right)\]
\[D = \left(\begin{array}{ccc}0 & 1 & b  \\a^2 & 0 & 0 \\ 0 & 0 & 1\end{array}\right)\]
\[E = \left(\begin{array}{ccc}t & 1 & 0  \\0 & 3 & 0 \\ h & 0 & 1\end{array}\right)\]

\vspace{0.4 cm}
\probl Donada la matriu
\[A = \left(\begin{array}{ccc}a+1 & a-1 & a  \\a-1 & a+1 & a \\ 0 & 0 & 1\end{array}\right)\]
\begin{enumerate}
\item Estudiau si $A$ �s o no �s diagonalitzable segons els valors del par�metre $a$.
\item Per $a=0$, calculau $A^n$.
\end{enumerate}
\end{document}  