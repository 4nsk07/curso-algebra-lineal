 \documentclass[12pt]{article}
\usepackage[catalan]{babel}
\usepackage{amsfonts,amssymb,amsmath,amsthm,hyperref,enumerate, cancel}
\usepackage[utf8]{inputenc}
\usepackage[T1]{fontenc}    
    
\advance\hoffset by -0.8in
\advance\textwidth by 1.6in
\advance\voffset by -0.9in
\advance\textheight by 1.8in
\parskip= 1 ex
\parindent = 10pt
\baselineskip= 13pt



\newcommand{\ZZ}{\mathbb{Z}}
\newcommand{\RR}{\mathbb{R}}
\newcommand{\NN}{\mathbb{N}}
\newcommand{\QQ}{\mathbb{Q}}
\renewcommand{\leq}{\leqslant}
\renewcommand{\geq}{\geqslant}

\newcounter{problemes}
\setcounter{problemes}{0}
\newcounter{punts}
\renewcommand{\thepunts}{\arabic{punts}}
\newcommand{\probl}{\addtocounter{problemes}{1}
\setcounter{punts}{0}
\medskip\noindent{{\bf \theproblemes) }}}
\newcommand{\problm}{\addtocounter{problemes}{1}
\setcounter{punts}{0}
\medskip\noindent{{\bf \theproblemes*) }}}

\newcommand{\punt}{\addtocounter{punts}{1}
\smallskip{{\emph{\thepunts) }}}}
\newcommand{\puntm}{\addtocounter{punts}{1}
\smallskip{{\emph{\thepunts*) }}}}

\newcount\problemes
\problemes=0

\def\probl{\advance\problemes by 1
\vskip 2ex\noindent{\bf \the\problemes \hbox{ } }}


\newcommand{\notimplies}{%
  \mathrel{{\ooalign{\hidewidth$\not\phantom{=}$\hidewidth\cr$\implies$}}}}





\begin{document}
\pagestyle{empty}

\parindent =0 pt
{\bf Algebra Lineal. Primer de Telemàtica. 
\hfill Primer parcial - 15 de Gener, 2018}

%Caotura pantalla
\vspace{1.0 cm}
\probl [3p] Considerau la matriu $A$ següent:
\[A=\left(\begin{array}{rrrr}1 & -2 & 5 & -3 \\ -1 & 0 & 0 & 1\\ 2 & 0 & 4 & 0\\ -3 & 2 &-5 & 0\end{array}\right)\]
\begin{enumerate}
\item Calcula el determinant de $A$ emprant les propietats dels determinants que consideris oportunes.
\item Calcula els setze adjunts de la matriu $A$ anterior
\item Empra els resultats anteriors per donar la inversa de $A$.
\end{enumerate}
\vspace{0.2 cm}

\probl [2p] Trobau en cada cas el vector o vectors de $\RR^3$ que compleixen
\begin{enumerate}
\item Té norma $\sqrt{2}$ i és perpendicular a $\vec{u} = (\sqrt{2},\sqrt{2},0)$ i a  $\vec{v} = (\sqrt{3},0, \sqrt{3})$ .
\item Té norma 1 i és perpendicular a $\vec{u} = (a^2, a^2, 1)$ i a l'eix positiu de les $X$.
\end{enumerate}

\vspace{0.2 cm}
\probl [2p] Definiu els següents conceptes:
\begin{enumerate}
\item Espai vectorial sobre un cos $\mathbb{K}$.
\item Subespai vectorial.
\end{enumerate}
\vspace{0.2 cm}

\probl [3p] Considerau el conjunt
\[S = \{(x,y,z)\in\RR^3: x+2y+z = 0;\ -x+y+z = 0\}\]
\begin{enumerate}
\item Demostrau que és un subespai vectorial de $\RR^3$.
\item Donau una base i dimensió del subespai.
\item Donau un vector que sigui perpendicular al subespai vectorial.
\end{enumerate}
\vspace{1.0 cm}
\newpage
{\bf Algebra Lineal. Primer de Telemàtica. 
\hfill Segon parcial}

\probl [5p] Donada la funció $f$ de $\mathbb{R}^3$  a $\mathbb{R}_{2}[t]$ determinada per $f(x,y,z)=(y+z)t^2 + (x+y)t+z$ en les bases canòniques respectives. Es demana:
\begin{enumerate}
\item Demostrar que és una aplicació lineal
\item Donar la matriu de $f$ en les bases canòniques.
\item Trobar el nucli de  $f$.
\item Trobar la imatge de $f$.
\item Donat el conjunt de vectors de l'espai vectoral $\mathbb{R}^3$ $U=\{(0,1,1), (1,0,1), (1,1,0)\}$ provau que en formen una base i donau la matriu de canvi de base de $U$ a la canònica d'aquest espai. 
\item Donau la matriu de canvi de base de la canònica a $U$ de $\mathbb{R}^3$
\item Donat el conjunt de polinomis de l'espai vectoral $\mathbb{R}_{2}[t]$ $V=\{t+1, t^2+1, t^2+t\}$ provau que en formen una base i donau la matriu de canvi de base de $V$ a la canònica d'aquest espai. 
\item Donau la matriu de canvi de base de la canònica a $V$ de $\mathbb{R}_{2}[t]$
\item Calculau l'expressió matricial de $f$ que comença a $\mathbb{R}^3$ en la base $U$ i acaba a  $\mathbb{R}_{2}[t]$ en la base $V$
\item Donau l'expressió analítica de la funció $f$ calculada a l'apartat anterior.
\end{enumerate}


%%exercici 4 full diagonalitzacio
\probl [2p] Provau que si $A$ és una matriu diagonalitzable i semblant a $B$, aleshores $B$ també és diagonalitzable. 
%%exercici 3 part2 full diagonalitzacio

\probl [3p] Donada la matriu
\[A=\left(\begin{array}{ccc}1 & a & a \\-1 & 1 & -1 \\1 & 0 & 2\end{array}\right)\]
\begin{enumerate}
\item Què ha de verificar el paràmetre $a$ perquè la matriu $A$ sigui diagonalitzable?
\item En els casos que diagonalitzi, cercau una base de vectors propis i la matriu diagonal.
\item Calculau $A^n$ per a tot $n$ natural emprant els resultats obtinguts.
\end{enumerate}

\vspace{2.5 cm}

\newpage

{\bf Algebra Lineal. Primer de Telemàtica. 
\hfill Tercer parcial }
%%exercici pagina 68 de portes logiques
\probl [3p] Per tal d'estar alerta davant les possibles invasions de Caminants Blancs que s'acosten al mur al Nord dels Set Regnes, el capità Jon Snow ha apostat quatre Corbs Negres que vigilaran zones estratègiques del Nord els quals, en cas de detectar perill, encendran una foguera ben gran per alertar els defensors del mur. Aquests, faràn sonar tres cops un corn que alerti del perill que s'acosta als campesins de les regions properes si s'encenen tres o bé les quatre fogueres. Si només s'activen dues de les fogueres, que facin sonar o no el corn és totalment indiferent i dependrà del vigilant de torn. El corn mai ha de sonar si només s'encen una o cap foguera per no asustar els campesins sense motiu. Finalment, per raons de seguretat, si s'activa només la darrera de les quatre fogueres, la més propera al mur es farà sonar si o si el Corn per estar segurs de l'amenaça inminent que s'acosta.

\begin{enumerate}
\item Descriure les variables d'una funció booleana $F$ (les que considereu necessàries) que compleixin totes aquestes condicions. 
\item Donar la seva taula de veritats.
\item Donar una expressió booleana de $F$.
\item Dibuixar el seu mapa de Karnaugh i una simplificació de l'expressió booleana anterior.
\end{enumerate}
\vspace{0.2 cm}


\probl [3p] Christian Grey és un magnat molt gelós de la seva nova al·lota, la senyoreta Anastasia Steele, i li imposa una sèrie de normes que ha de complir i firmar en un document per poder dur la seva relació envant. Les normes es refereixen als tres aspectes següents:
\begin{itemize}
\item x: prendre vi reserva
\item y: dur roba elegant per sortir
\item z: conduir el cotxe d'en Christian
\end{itemize}
Les normes que li imposa son que 
\begin{itemize}
\item mai prendrà alcohol de reserva si alhora ha de conduir,
\item o bé prendrà vi reserva en cas de anar elegant per sortir o bé conduirà el cotxe si va elegant i no ha begut vi.
\end{itemize}

N'Anastasia accedeix a la segona però es rebel·la contra la primera i diu que mai la complirà, cosa que en Christian acaba acceptant. Donades les restriccions anteriors, 
%$f(x,y,z)=\overline{(\bar{x}y)}(xy+\overline{\bar{x}yz})$
\begin{enumerate}
\item Donau la funció booleana $f(x,y,z)$ que resumeix les condicions que acorden la parella.
\item Simplificau-la per donar-ne la forma canònica conjuntiva.
\item Simplificau-la per donar-ne la  forma canònica disjuntiva.
\end{enumerate}
\vspace{0.2 cm}

\probl [4p] Un dels jocs favorits del Dr. Sheldon Cooper per resoldre conflictes i prende decisions és el de pedra, paper, tisores, llangardaix, Spock. En aquest joc, es segueixen una sèrie de regles bàsiques: 

\vspace{0.2 cm}

El paper cobreix la pedra, aquesta esclafa el llangardaix, aquest enverina n'Spock, aquest  vaporitza la pedra, aquesta romp les tisores que a la seva vegada decapiten el llangardaix, aquest té gana i es menja el paper, el qual desprestigia n'Spock, que a la seva vegada romp les tisores, i com no, aquestes tallen el paper.  

\begin{enumerate}
\item Dibuixau un graf dirigit on es vegin els cinc possibles moviments del joc i la relació que s'estableix entre cadascún d'ells. 
\item Donau el node o nodes de major grau del graf i indicau-ne el valor.
\item El graf que heu dibuixat és un arbre? Perquè?
\item El graf que heu dibuixat és connex? Perquè? Quantes components connexes té?
\item Digau si conté un circuit i/o recorregut eulería justificant la vostra resposta. 
\item Digau si conté un circuit i/o recorregut hamiltonià justificant la vostra resposta. 
\item Donat un graf $G=(V,E)$ s'anomena graf complementari al graf que conté els mateixos vèrtexos que $G$ però les arestes que no contenía el graf original. Dibuixau i donau el graf complementari de $G$.
\item Son isomorfs el graf original $G$ i el seu complementari $G'$? En cas afirmatiu donau un isomorfisme, i en cas negatiu explicau perquè no ho poden ser. 
\item Escriviu amb paraules les regles del joc que resulten del graf complementari de $G$ de forma original (per exemple, n'Spock es menja el llangardaix). 
\item Dibuixau el graf original ara no dirigit i assignau a cada aresta el pes equivalent a la suma de vocals entre les dues paraules de cada node que uneix l'aresta (per exemple, l'aresta que uneix pedra amb Spock haurà de tenir pes 3). Donau-ne un arbre recobridor mínim. 
\end{enumerate}

\textbf{Segurament el curs podría haver estat més complet i amb més contingut, exercicis i demostracions però mai hagues pogut ser més divertit que com ha estat. Molta sort i fins el cuatrimestre que ve!}

\end{document}  