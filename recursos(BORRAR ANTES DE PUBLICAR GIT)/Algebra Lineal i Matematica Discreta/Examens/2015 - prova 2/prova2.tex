\documentclass[14p,spanish]{article}
\usepackage[spanish]{babel}
\usepackage[ansinew]{inputenc}
\usepackage[T1]{fontenc}
\usepackage{graphicx}
\usepackage{multicol}
\usepackage{longtable}
\usepackage{array}
\usepackage{multirow}
\usepackage{geometry}                		
\geometry{letterpaper}                   		
\usepackage{graphicx}
\usepackage{amssymb}
\usepackage{color}


\setlength{\textwidth}{16cm}
\setlength{\textheight}{24cm}
\setlength{\oddsidemargin}{-0.3cm}
\setlength{\topmargin}{-1.3cm}


\newcommand{\sC}{{\cal C}}
\newcommand{\sF}{{\cal F}}
\newcommand{\sL}{{\cal L}}
\newcommand{\sU}{{\cal U}}
\newcommand{\sX}{{\cal X}}
\newcommand{\eop}{{\Box}}

\newcommand{\ar}{A^{(r)}}
\newcommand{\HH}{{\bf H}}
\newcommand{\sS}{{\cal S}}
\newcommand{\Img}{\mbox{Img}}

\def\N{I\!\!N}
\def\R{I\!\!R}
\def\Z{Z\!\!\!Z}
\def\Q{O\!\!\!\!Q}
\def\C{C\!\!\!\!I}


\newcount\problemes
\problemes=0

\def\probl{\advance\problemes by 1
\vskip 2ex\noindent{\bf \the\problemes \hbox{ } }}

\graphicspath{ {im/} }


\newcommand{\notimplies}{%
  \mathrel{{\ooalign{\hidewidth$\not\phantom{=}$\hidewidth\cr$\implies$}}}}





\begin{document}
\pagestyle{empty}

\parindent =0 pt
{\bf Algebra Lineal. Primer de Telem�tica. 
\hfill Segon parcial - 19 de Desembre, 2015}
 

\vspace{0.6 cm}
\probl (3p) Considera la seg�ent matriu:
\[A = \left(\begin{array}{ccc}1 & a & a \\-1 & 1 & -1 \\1 & 0 & 2\end{array}\right)\]
\begin{enumerate}
\item (1p) Qu� ha de verificar el par�metre $a\in\mathbb{R}$ per tal que la matriu $A$ sigui diagonalitzable sobre els reals? 
\item (1p) Quan ho sigui, trobau la seva forma diagonal, una matriu de canvi de base.
\item (1p) Calculau $A^n$ per a qualssevol nombre $n$ natural. 
\end{enumerate}
%%RESULETO EN REL 6



\vspace{0.4cm}
\probl (4p) Sigui $f:\mathbb R^3\longrightarrow \mathbb R^2$ definida per $f(x,y,z) = (3x+2y-4z, x-5y+3z)$ 
\begin{enumerate}
\item (1p) Demostrau que l'aplicaci� $f$ �s lineal. 
\item (1p) Trobau el nucli i la imatge de $f$. Discutiu quin tipus d'aplicaci� �s (i.e. monomorfisme, epimorfisme, isomorfisme, automorfisme,...)
\item (0.5p) Calculau la matriu de $f$ respecte de les bases can�niques respectives.
\item (1.5p) Calculau la matriu de $f$ respecte de les bases $B_{\mathbb R^3} =\{(1,1,1),(1,1,0),(1,0,0)\}$ i $B_{\mathbb R^2}=\{(1,1), (0,1)\}$.
\end{enumerate} 

\textbf{Heu d'entregar aquest exercici com a tard el d�a del Segon Parcial (19 de Desembre de 2015). }

\vspace{0.6 cm}
\end{document}  