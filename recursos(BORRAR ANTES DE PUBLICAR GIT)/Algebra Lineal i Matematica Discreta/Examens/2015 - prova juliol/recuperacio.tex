\documentclass[12p,spanish]{article}
\usepackage[spanish]{babel}
\usepackage[ansinew]{inputenc}
\usepackage[T1]{fontenc}
\usepackage{graphicx}
\usepackage{multicol}
\usepackage{longtable}
\usepackage{array}
\usepackage{multirow}
\usepackage{geometry}                		
\geometry{letterpaper}                   		
\usepackage{graphicx}
\usepackage{amssymb}
\usepackage{color}


\setlength{\textwidth}{16cm}
\setlength{\textheight}{24cm}
\setlength{\oddsidemargin}{-0.3cm}
\setlength{\topmargin}{-1.3cm}


\newcommand{\sC}{{\cal C}}
\newcommand{\sF}{{\cal F}}
\newcommand{\sL}{{\cal L}}
\newcommand{\sU}{{\cal U}}
\newcommand{\sX}{{\cal X}}
\newcommand{\eop}{{\Box}}

\newcommand{\ar}{A^{(r)}}
\newcommand{\HH}{{\bf H}}
\newcommand{\sS}{{\cal S}}
\newcommand{\Img}{\mbox{Img}}

\def\N{I\!\!N}
\def\R{I\!\!R}
\def\Z{Z\!\!\!Z}
\def\Q{O\!\!\!\!Q}
\def\C{C\!\!\!\!I}


\newcount\problemes
\problemes=0

\def\probl{\advance\problemes by 1
\vskip 2ex\noindent{\bf \the\problemes \hbox{ } }}

\graphicspath{ {im/} }


\newcommand{\notimplies}{%
  \mathrel{{\ooalign{\hidewidth$\not\phantom{=}$\hidewidth\cr$\implies$}}}}





\begin{document}
\pagestyle{empty}

\parindent =0 pt
{\bf Algebra Lineal. Primer de Telem�tica. 
\hfill Examen de recuperaci� - 27 de Juny, 2016}

\vspace{1.5 cm}
\probl [1p] Donats els vectors $\vec u = (3,0,0), \vec v = (0,-2,-1)$ i $\vec w = a\vec u + b\vec v$, quina condici� han de complir els escalars $a$ i $b$ per tal de que satisfaguin alhora:
\begin{itemize}
\item $\vec w$ sigui ortogonal al vector $(1,1,1)$.
\item $\vec w$ sigui unitari.
\end{itemize}
\vspace{0.6 cm}

\probl [1p] Considerau els vectors del conjunt $C=\{(1,0,0),(1,1,0),(1,1,1)\}$.
\begin{enumerate}
\item Demostrau que formen una base de $\mathbb R^3$.
\item Trobau respecte d'aquesta base les coordenades del vector $(5,1,-3)$.
\end{enumerate}
\vspace{0.6 cm}

\probl [3p] Un endomorfisme $f$ de $\mathbb{R}^3$ est� determinat per $f(x,y,z)=(x+3z, x-2y, -2z)$ en la base can�nica. Es demana
\begin{enumerate}
\item Trobar el nucli i la imatge de $f$. Classificau el morfisme $f$ segons els resultats obtinguts.
\item Trobar la matriu de $f$ en aquesta base
\item Trobar la matriu de $f$ en la base $V$ constituida pels vectors $v_1=(1,1,1), v_2=(1,1,0), v_3=(1,0,0)$
\item L'expressi� anal�tica de $f$ en aquesta base $V$
\end{enumerate}
\vspace{0.6 cm}

\probl [3p] Donada la matriu
\[A=\left(\begin{array}{ccc}2 & -2 & 6 \\0 & a & 4-a \\0 & a & -a\end{array}\right)\]
\begin{enumerate}
\item Trobau els valors  de $a$ per els quals $\lambda = 4$ �s un valor propi de $A$.
\item Per $a = 1$, donau si �s possible una matriu $P$ de vectors propis i la seva matriu diagonal $D$ corresponent.
\item Per $a = 1$, calculau $A^n$ per a tot $n\geq 0$. 
\end{enumerate}

\vspace{0.6 cm}

\probl [2p] Donada la funci� booleana $f(x,y,z)={(\bar{x}y)}(x+\overline{xy\bar{z}})$
\begin{enumerate}
\item Donau la forma can�nica disjuntiva
\item Donau la taula de veritat de la funci�
\item Dibuixau el seu mapa de Carnaugh i emprau-lo per donar-ne una simplificaci�. 
\end{enumerate}
\vspace{0.6 cm}

\vspace{0.6 cm}
\end{document}  