\documentclass[12p,spanish]{article}
\usepackage[spanish]{babel}
\usepackage[ansinew]{inputenc}
\usepackage[T1]{fontenc}
\usepackage{graphicx}
\usepackage{multicol}
\usepackage{longtable}
\usepackage{array}
\usepackage{multirow}
\usepackage{geometry}                		
\geometry{letterpaper}                   		
\usepackage{graphicx}
\usepackage{amssymb}
\usepackage{color}


\setlength{\textwidth}{16cm}
\setlength{\textheight}{24cm}
\setlength{\oddsidemargin}{-0.3cm}
\setlength{\topmargin}{-1.3cm}


\newcommand{\sC}{{\cal C}}
\newcommand{\sF}{{\cal F}}
\newcommand{\sL}{{\cal L}}
\newcommand{\sU}{{\cal U}}
\newcommand{\sX}{{\cal X}}
\newcommand{\eop}{{\Box}}

\newcommand{\ar}{A^{(r)}}
\newcommand{\HH}{{\bf H}}
\newcommand{\sS}{{\cal S}}
\newcommand{\Img}{\mbox{Img}}

\def\N{I\!\!N}
\def\R{I\!\!R}
\def\Z{Z\!\!\!Z}
\def\Q{O\!\!\!\!Q}
\def\C{C\!\!\!\!I}


\newcount\problemes
\problemes=0

\def\probl{\advance\problemes by 1
\vskip 2ex\noindent{\bf \the\problemes \hbox{ } }}

\graphicspath{ {im/} }


\newcommand{\notimplies}{%
  \mathrel{{\ooalign{\hidewidth$\not\phantom{=}$\hidewidth\cr$\implies$}}}}





\begin{document}
\pagestyle{empty}

\parindent =0 pt
{\bf Algebra Lineal. Primer de Telem�tica. 16/17
\hfill Examen de recuperaci� - 26 de Juny, 2017}

\probl [2p] Donats els vectors no nuls $\vec u = (u_1,u_2,u_3), \vec v = (v_1, v_2,v_3)$ demostrau les seg�ents proposicions
\begin{itemize}
\item $\vec u \cdot \vec v = \vec v \cdot \vec u$
\item $\vec u \cdot \vec v = 0\Longleftrightarrow \alpha = \frac{\pi}{2}$.
\item $||\vec u + \vec v ||\leq ||\vec u|| + ||\vec v||$
\end{itemize}

\probl [2.5p] Un endomorfisme $f$ de $\mathbb{R}^3$ est� determinat per $f(x,y,z)=(2y+z, x-4y, 3x)$ en la base can�nica. Es demana
\begin{enumerate}
\item Trobar el nucli i la imatge de $f$. Classificau el morfisme $f$ segons els resultats obtinguts.
\item Trobar la matriu de $f$ en aquesta base
\item Trobar la matriu de $f$ en la base $V$ constituida pels vectors $v_1=(1,1,1), v_2=(1,1,0), v_3=(1,0,0)$
\item L'expressi� anal�tica de $f$ en aquesta base $V$
\end{enumerate}

\probl [2.5p] Donada la matriu
\[A=\left(\begin{array}{ccc}2 & a & a \\a & 2 & a \\a & a & 2\end{array}\right)\]
%det = 8 + 2a^3-6a^2
\begin{enumerate}
\item Trobau els valors  propis de $A$ en funci� del par�metre $a$ 
\item Trobau els vectors propis de $A$ en funci� del par�metre $a$ 
\item Per quins valors de $A$ �s la matriu diagonalitzable? Per aquests casos donau la matriu diagonal i la matriu de canvi que base pertinent.
\item En els casos en que la matriu diagonalitzi, donau una expressi� per $A^n$ en funci� del par�metre $a$.
\end{enumerate}

\probl [3p] La vora externa dels planetes de la Galaxia Llunyana est� conformada per cinc planetes connectats en forma de pent�gon: Alderaan, que dista 1 any llum de Dantooine, que dista 20 anys llum de Naboo, que dista 15 anys llum de Tatooine, que dista 1 any llum de Utapau, el qual dista 2 anys llums del primer planeta, Alderaan. 
El nucli profund dels planetes de la mateixa galaxia est� conformada per cinc planetes m�s connectats en forma d'estrella de cinc puntes: Coruscant, que dista 4 anys llum de Mustafar, que dista 9 anys llum de Hoth, que dista 1 any llum d'Endor, que dista 2 anys llum de Polis Massa, el qual dista 1 any llums del primer planeta, Coruscant. 
Finalment, cada planeta de l'orbita exterior es connecta amb exactament un planeta de l'�rbita interior: Alderaan amb Coruscant a una dist�ncia de 9 anys llum, Dantooine amb Endor a 7 anys llum, Naboo amb Mustafar a 2 anys llum, Tatooine amb Polis Massa a 4 anys llum i Utapau amb Hoith a 1 any llum.

\begin{enumerate}
\item Dibuixau un graf on es vegi la connexi� entre els diferents planetes de la galaxia indicant a cada node el planeta en q�esti� i a cada aresta la dist�ncia entre ells.
\item Donau la matriu associada al graf anterior (on hi hagi un zero no cal que poseu cap n�mero).
\item Trobau el planeta o planetes de major grau del graf i donau un recorregut simple que comenci en el planeta indicat i passi per tots els altres. Donau tamb� si �s possible un cicle que passi per tots els planetes. 
\item Donau un arbre recobridor m�nim, �s a dir, per quines v�es intergal�ctiques posarieu la xarxa de fibra �ptica intergal�ctica per tal de gastar el m�nim de fibra �ptica possible (que va un poc cara segons ens diuen els t�cnics de Darth Vaderf�nica).
\item �s possible fer un recorregut que passi per tots els planetes de la galaxia sense repetir-ne cap? Justificau la resposta amb alg�n dels resultats que heu vist a classe (no val dir si o no sense m�s). En cas afirmatiu, marcau-ne un, i en cas negatiu digau perqu� �s impossible trobar tal cam�.  
\end{enumerate}
\textbf{Que la for�a us acompanyi!}
\vspace{0.6 cm}
\end{document}  