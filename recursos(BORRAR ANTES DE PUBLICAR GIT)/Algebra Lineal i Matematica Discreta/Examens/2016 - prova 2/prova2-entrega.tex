\documentclass[14p,spanish]{article}
\usepackage[spanish]{babel}
\usepackage[ansinew]{inputenc}
\usepackage[T1]{fontenc}
\usepackage{graphicx}
\usepackage{multicol}
\usepackage{longtable}
\usepackage{array}
\usepackage{multirow}
\usepackage{geometry}                		
\geometry{letterpaper}                   		
\usepackage{graphicx}
\usepackage{amssymb}
\usepackage{color}


\setlength{\textwidth}{16cm}
\setlength{\textheight}{24cm}
\setlength{\oddsidemargin}{-0.3cm}
\setlength{\topmargin}{-1.3cm}


\newcommand{\sC}{{\cal C}}
\newcommand{\sF}{{\cal F}}
\newcommand{\sL}{{\cal L}}
\newcommand{\sU}{{\cal U}}
\newcommand{\sX}{{\cal X}}
\newcommand{\eop}{{\Box}}

\newcommand{\ar}{A^{(r)}}
\newcommand{\HH}{{\bf H}}
\newcommand{\sS}{{\cal S}}
\newcommand{\Img}{\mbox{Img}}

\def\N{I\!\!N}
\def\R{I\!\!R}
\def\Z{Z\!\!\!Z}
\def\Q{O\!\!\!\!Q}
\def\C{C\!\!\!\!I}


\newcount\problemes
\problemes=0

\def\probl{\advance\problemes by 1
\vskip 2ex\noindent{\bf \the\problemes \hbox{ } }}

\graphicspath{ {im/} }


\newcommand{\notimplies}{%
  \mathrel{{\ooalign{\hidewidth$\not\phantom{=}$\hidewidth\cr$\implies$}}}}





\begin{document}
\pagestyle{empty}

\parindent =0 pt
{\bf Algebra Lineal. Primer de Telem�tica. 
\hfill Segon parcial - 21 de Desembre, 2016}
 

\vspace{0.6 cm}
\probl (2p) Amb motiu de les super notes tretes al primer quatrimestre del grau de telem�tica, na Paula, na Margalida, en Sergi i en Pau, que son uns fiesteros com cap altre, ens preparen una Maaacrofesta. 

En Marcos i n'�ngel son dos pagesos enrrollats i ens deixen les seves parcel�les per conrear remolatxa i canya de sucre. 
\begin{itemize}
\item La parcela d'en Marcos t� $40 Ha$ de terra conreable i disposa d'un dip�sit de $500 m^3$ d'aigua potable, mentres que n'�ngel, a la seva gran mansi� t� $90 Ha$ de s�l f�rtil i disposa de $1200 m^3$ d'aigua. 
\item El expert en alcohol, en Ramon ens diu que la remolatxa consumeix $3 m^3$ d'aigua per $Ha$ conreada i ens permet elaborar 7 ampolles de ron blanc, amb les quals preparar 10 mojitos amb cada una, mentres que la canya de sucre consumeix $2 m^3$ d'aigua per $Ha$ conreada i ens permet elaborar 10 ampolles de ron negre, cada una de les quals ens permet elaborar 5 cubates. 
\item A m�s, com que ve el profe a la festa, sabem que necessitarem com a m�xim $5600$ mojitos i $3000$ cubates 
\item No volem aprofitar-nos de la bondat dels dos pagesos, aix� que acordam que les dues parcel�les tinguin el mateix percentatge conreat (per exemple, les dues parcel�les conreades al 75$\%$).
\end{itemize}

Amb tota aquesta informaci�:
\begin{enumerate}
\item Plantejau el problema de programaci� lineal pertinent que maximitzi el nombre de mojitos i cubates que els barmans Jorge i Lydia ens podran preparar a la festa.
\item Expressau-lo en forma est�ndard 
\item Resoleu-lo pel m�tode del s�mplex i digueu quantes hect�res de remolatxa i de canya de sucre hem de plantar a cadascun dels camps respectius d'en Marcos i de n'�ngel. 
\end{enumerate}


PD: qualssevol semblan�a amb la realitat d'aquest problema �s fruit de l'atzar, aix� que no us ho pregueu al peu de la lletra i us presenteu a classe amb 5600 mojitos i 3000 cubates, que ens coneixem. 

\textbf{Heu d'entregar aquest exercici com a tard el d�a del Segon Parcial (21 de Desembre de 2016). }

\vspace{0.6 cm}
\end{document}  