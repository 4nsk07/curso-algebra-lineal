\documentclass[11pt, oneside]{article}   	% use "amsart" instead of "article" for AMSLaTeX format
\usepackage{geometry}                		% See geometry.pdf to learn the layout options. There are lots.
\geometry{letterpaper}                   		% ... or a4paper or a5paper or ... 
%\geometry{landscape}                		% Activate for rotated page geometry
%\usepackage[parfill]{parskip}    		% Activate to begin paragraphs with an empty line rather than an indent
\usepackage{graphicx}				% Use pdf, png, jpg, or eps§ with pdflatex; use eps in DVI mode
								% TeX will automatically convert eps --> pdf in pdflatex		
\usepackage{amssymb}
\usepackage[spanish]{babel}
\usepackage[latin1]{inputenc}
%SetFonts

%SetFonts


\title{Pr�ctica 3}

\begin{document}
\maketitle
%\section{}
%\subsection{}

En el espacio vectorial $\mathbb R^3 [x]$ de los polinomios de grado menor o igual que 3, consideramos los subespacios
\[F = \{ P(x) : \ P(0) = P(1) = P'(1/2) = P''' (0) = 0\}\]
\[G = [x+1, x, x-1]\]
\begin{enumerate}
\item Demostrar que $F$ est� formado por los polinomios $a_0+a_1x+a_2x^2+a_3x^3$ que verifican que $a_0=a_3 = 0$ y $a_1=-a_2$.
\item Hallad una base y la dimensi�n de $F$ y de $G$.
\item �Son los polinomios $x^2-5x+2$ y $3x-4$ elementos de $G$? En caso afirmativo, expresadlo como combinaci�n lineal de $[x+1, x, x-1]$.
\item Describid una base y la dimensi�n de los elementos de $F\cap G$. 
\end{enumerate}

\end{document}  