\documentclass[aspectratio=169]{beamer}
\usepackage[spanish]{babel}
\usepackage[latin1]{inputenc}
\usepackage{multicol} % indice en 2 columnas
\usepackage{centernot}
\usepackage{amsmath}% http://ctan.org/pkg/amsmath

\newcommand{\notimplies}{%
  \mathrel{{\ooalign{\hidewidth$\not\phantom{=}$\hidewidth\cr$\implies$}}}}


\usetheme{Warsaw}
%\usecolortheme{crane}
\useoutertheme{shadow}
\useinnertheme{rectangles}

\setbeamertemplate{navigation symbols}{} % quitar simbolitos



\title[Tema 1 - Matrices, sistemas y determinantes]{C\'{a}lculo matricial - Ejercicios II}
\subtitle{Estudios de Ingenier\'ia}
\author[James Bond]{
James Bond%$^{1}$  \and E. Eva$^{2}$ \and S. Serpiente$^{3}$
}
\date{}

\AtBeginSection{
\begin{frame}
  \begin{multicols}{2}
  \tableofcontents[currentsection]   
\end{multicols}
\end{frame}
}

\AtBeginSubsection{
\begin{frame}
  \begin{multicols}{2}
  \tableofcontents[currentsection,currentsubsection]
\end{multicols}
\end{frame}
}



%empieza aqui


\begin{document} 

\frame{\titlepage}

\begin{frame}
  \frametitle{Ejercicios de Operaciones Matriciales}
  \begin{block}{Ejercicio 1}
Halla la forma escalonada y escalonada reducida por filas de la matriz
\[
A= \left(\begin{array}{rrrr}
1&2&1&0\\
1&3&2&1
\end{array}\right)
\]
  \end{block}
\end{frame}

\begin{frame}
  \frametitle{Ejercicios de Operaciones Matriciales}
  \begin{block}{Ejercicio 2}
Halla la forma escalonada y escalonada reducida por filas de la matriz
\[
A= \left(\begin{array}{rrr}
1&2&1\\
1&3&2\\
1&1&0
\end{array}\right)
\]
  \end{block}
\end{frame}


\begin{frame}
  \frametitle{Ejercicios de Operaciones Matriciales}
  \begin{block}{Ejercicio 3}
Halla la forma escalonada y escalonada reducida por filas de la matriz
\[
A= \left(\begin{array}{rrr}
2&4&-3\\
-1&-2&3\\
-1&-2&-3
\end{array}\right)
\]
  \end{block}
\end{frame}

\begin{frame}
  \frametitle{Ejercicios de Operaciones Matriciales}
  \begin{block}{Ejercicio 4}
Halla la forma escalonada y escalonada reducida por filas de la matriz
\[
A= \left(\begin{array}{rrrr}
1&2&1&2\\
2&1&-5&-3\\
-3&1&2&7
\end{array}\right)
\]
  \end{block}
\end{frame}

\begin{frame}
  \frametitle{Ejercicios de Operaciones Matriciales}
  \begin{block}{Ejercicio 5}
Halla el rango de la matriz
\[
A= \left(\begin{array}{rrr}
-1&1&-2\\
1&1&0\\
2&1&1
\end{array}\right)
\]
  \end{block}
\end{frame}


\begin{frame}
  \frametitle{Ejercicios de Operaciones Matriciales}
  \begin{block}{Ejercicio 6}
Halla el rango de la matriz
\[
A= \left(\begin{array}{rrrrr}
-1&2&3&4&5\\
1&2&1&3&2\\
0&4&4&7&7\\
\end{array}\right)
\]
  \end{block}
\end{frame}


\begin{frame}
  \frametitle{Ejercicios de Operaciones Matriciales}
  \begin{block}{Ejercicio 7}
Halla el rango de la matriz
\[
A= \left(\begin{array}{rrrrr}
1&-2&1&0&2\\
0&1&2&-1&1\\
1&2&0&-2&1\\
2&1&3&-3&4\\
3&-2&2&-2&5\\
\end{array}\right)
\]
  \end{block}
\end{frame}


\begin{frame}
  \frametitle{Ejercicios de Operaciones Matriciales}
  \begin{block}{Ejercicio 8}
Halla la inversa de la siguiente matriz:
\[
A= \left(\begin{array}{rrr}
1&0&0\\
2&1&0\\
1&-1&3
\end{array}\right)
\]
  \end{block}
\end{frame}

\begin{frame}
  \frametitle{Ejercicios de Operaciones Matriciales}
  \begin{block}{Ejercicio 9}
Halla la inversa de la siguiente matriz:
\[
A= \left(\begin{array}{rrr}
-2&1&3\\
0&-1&1\\
1&2&0
\end{array}\right)
\]
  \end{block}
\end{frame}

\begin{frame}
  \frametitle{Ejercicios de Operaciones Matriciales}
  \begin{block}{Ejercicio 10}
Halla la inversa de la siguiente matriz:
\[
A= \left(\begin{array}{rrr}
1&2&0\\
0&1&-1\\
0&0&2
\end{array}\right)
\]
  \end{block}
\end{frame}

\end{document}