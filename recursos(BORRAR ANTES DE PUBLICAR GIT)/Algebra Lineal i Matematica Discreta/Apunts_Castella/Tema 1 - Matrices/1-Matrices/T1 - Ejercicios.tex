\documentclass[aspectratio=169]{beamer}
\usepackage[spanish]{babel}
\usepackage[latin1]{inputenc}
\usepackage{multicol} % indice en 2 columnas
\usepackage{centernot}
\usepackage{amsmath}% http://ctan.org/pkg/amsmath

\newcommand{\notimplies}{%
  \mathrel{{\ooalign{\hidewidth$\not\phantom{=}$\hidewidth\cr$\implies$}}}}


\usetheme{Warsaw}
%\usecolortheme{crane}
\useoutertheme{shadow}
\useinnertheme{rectangles}

\setbeamertemplate{navigation symbols}{} % quitar simbolitos



\title[Tema 1 - Matrices, sistemas y determinantes]{C\'{a}lculo matricial - Ejercicios I}
\subtitle{Estudios de Ingenier\'ia}
\author[James Bond]{
James Bond%$^{1}$  \and E. Eva$^{2}$ \and S. Serpiente$^{3}$
}

\date{}

\AtBeginSection{
\begin{frame}
  \begin{multicols}{2}
  \tableofcontents[currentsection]   
\end{multicols}
\end{frame}
}

\AtBeginSubsection{
\begin{frame}
  \begin{multicols}{2}
  \tableofcontents[currentsection,currentsubsection]
\end{multicols}
\end{frame}
}



%empieza aqui


\begin{document} 

\frame{\titlepage}

\begin{frame}
  \frametitle{Ejercicios de Matrices}
  \begin{block}{Ejercicio 1}
  Consideremos las matrices
\[
A= \left(\begin{array}{rrr}
0&1&-2\\
2&3&-1\\
1&-1&5
\end{array}\right) 
B= \left(\begin{array}{rrrr}
1&-1&2&1\\
2&-2&2&-2\\
-1&2&1&2
\end{array}\right)
C= \left(\begin{array}{r}
2\\0\\1\\-4
\end{array}\right)
\]
Realizad las operaciones siguientes:
\[a)A\cdot B,\ \ b)B\cdot C,\ \ c)B^t,\ \ d)B^t\cdot A,\ \ e)C^t\cdot B^t\ \ \]
  \end{block}
\end{frame}


\begin{frame}
  \frametitle{Ejercicios de Matrices}
  \begin{block}{Ejercicio 2}
Escribid la matrix 3x4 que tiene por entrada $(i,j)$ el elemento
\[a_{ij} = \frac{(-1)^{i+j}}{i+j}\]

  \end{block}
\end{frame}

\begin{frame}
  \frametitle{Ejercicios de Matrices}
  \begin{block}{Ejercicio 3}
Escribid la matrix (n+1)x(n+1) que tiene por entrada $(i,j)$ el elemento
\[a_{ij} = \left\{\begin{array}{ccc}0 & \mathrm{si} & i>j \\1 & \mathrm{si} & i=j \\k^{j-i} & \mathrm{si} & j>i\end{array}\right.\]
donde $k$ es un n�mero real cualquiera.
  \end{block}
\end{frame}

\begin{frame}
  \frametitle{Ejercicios de Matrices}
  \begin{block}{Ejercicio 4}
Dada la matriz
\[A= \left(\begin{array}{rr}
0&1\\0&0
\end{array}\right)\]
hallad todas las matrices cuadradas de orden 2 tales que $AX=0$
  \end{block}
\end{frame}


\begin{frame}
  \frametitle{Ejercicios de Matrices}
  \begin{block}{Ejercicio 5}
Considerad las matrices
\[A= \left(\begin{array}{rr}
0&1\\0&1
\end{array}\right)
B= \left(\begin{array}{rr}
-1&-1\\0&0
\end{array}\right)\]
Demostrad que $(A+B)^2\neq A^2+2AB+B^2$, pero en cambio que $(A+B)^3 = A^3+3A^2B+3AB^2+B^3$.  \end{block}
\end{frame}


\begin{frame}
  \frametitle{Ejercicios de Matrices}
  \begin{block}{Ejercicio 6}
Hallad matrices $A$ y $B$ tales que cumplan las dos ecuaciones
\[4A+2B= \left(\begin{array}{rrr}
3&4&2\\2&1&8
\end{array}\right)\]
\[3A+B= \left(\begin{array}{ccc}
3/2&1&0\\2&1/2&5
\end{array}\right)\]
  \end{block}
\end{frame}

\begin{frame}
  \frametitle{Ejercicios de Matrices}
  \begin{block}{Ejercicio 7}
Sean $a,b,c\in \mathbb R$ y $A$ la matriz dada por \[
A= \left(\begin{array}{ccc}
a&a+b&a-c\\
a-b&b&b-c\\
a+b-2&c-b&c
\end{array}\right)\]
Qu� tienen que valer los par�metros para que $A$ sea
\begin{enumerate}
\item Triangular superior
\item Triangular inferior
\item Sim�trica
\end{enumerate}
  \end{block}
\end{frame}



\begin{frame}
  \frametitle{Ejercicios de Matrices}
  \begin{block}{Ejercicio 8}
Dada la matriz
\[A= \left(\begin{array}{rr}
1&0\\1&1
\end{array}\right)\]
Calculad el valor de 
\[A+A^2+\cdots+A^n\] 
para todo valor $n\geq1$
 \end{block}
\end{frame}




\begin{frame}
  \frametitle{Ejercicios de Matrices}
  \begin{block}{Ejercicio 9}
Sea $A$ una matriz cuadrada de orden $n$. $B$ se llama una ra�z cuadrada de $A$ si $B^2 = A$.
\begin{enumerate}
\item Halla  tres ra�ces cuadradas diferentes de $I_2$.
\item Demuestra que la matriz $ \left(\begin{array}{rr}
0&1\\0&0
\end{array}\right)$ no tiene ra�ces cuadradas.
\end{enumerate}
  \end{block}
\end{frame}


\begin{frame}
  \frametitle{Ejercicios de Matrices}
  \begin{block}{Ejercicio 10}
Halla todas las matrices reales que conmutan con la matriz
\[A= \left(\begin{array}{rrr}
0&0&0\\1&0&0\\1&1&0
\end{array}\right)\]
 \end{block}
\end{frame}



\begin{frame}
  \frametitle{Ejercicios de Matrices}
  \begin{block}{Ejercicio 11}
Halla las potencias $n$-�simas de las matrices 
\[A= \left(\begin{array}{rrr}
0&1&0\\0&0&1\\0&0&0
\end{array}\right)
B= \left(\begin{array}{rrr}
1&1&1\\1&1&1\\1&1&1
\end{array}\right)\]
 \end{block}
\end{frame}

\begin{frame}
  \frametitle{Ejercicios de Matrices}
  \begin{block}{Ejercicio 12}
Sean $A,B\in M_n(\mathbb R)$ dos matrices tales que $A$ es sim�trica y $B$ es antisim�trica. Demostrad que $AB+BA$ es antisim�trica y que $AB-BA$ es sim�trica
 \end{block}
\end{frame}

\begin{frame}
  \frametitle{Ejercicios de Matrices}
  \begin{block}{Ejercicio 13}
Sea $A$ una matriz cuadrada de orden $n$. Demostrad que:
\begin{enumerate}
\item $A+A^t$ es sim�trica
\item $A-A^t$ es antisim�trica
\item $A $ se puede poner siempre como suma de una matriz sim�trica y una antisim�trica 
\end{enumerate}
 \end{block}
\end{frame}

\begin{frame}
  \frametitle{Ejercicios de Matrices}
  \begin{block}{Ejercicio 14}
Sean $A$ y $B$ matrices cuadradas de orden $n$ de modo que $B$ es sim�trica. Demostrad que:
 \begin{enumerate}
\item $AA^t$ es sim�trica
\item $ABA^t$ es sim�trica
\item Si $A$ es antisim�trica, entonces $A^2$ es sim�trica.
\item Si $A^2=0$ entonces $A(A+I_n)^i=A$ para todo $i=0,1,2,3,\cdots$
\end{enumerate}
 \end{block}
\end{frame}

\begin{frame}
  \frametitle{Ejercicios de Matrices}
  \begin{block}{Ejercicio 15}
Una matriz $A=(a_{ij})\in M_n(\mathbb R)$ se llama estoc�stica si
 \begin{itemize}
\item Todos sus coeficientes son no negativos, es decir $a_{ij}\geq 0$ para todos $i,j=1,2,\cdots,n$
\item La suma de los coeficientes de cada fila vale 1, es decir $\sum_{j=1}^n a_{ij} = 1\ \forall \ i=1,\cdots,n$
\end{itemize}
y doblemente estoc�stica si adem�s la suma de los coeficientes de cada columna tambi�n vale 1, es decir  $\sum_{i=1}^n a_{ij} = 1\ \forall \ j=1,\cdots,n$
 \begin{enumerate}
 \item Da un ejemplo de matriz $A\in M_4(\mathbb R)$ estoc�stica
 \item Da un ejemplo de matriz $A\in M_4(\mathbb R)$ doblemente estoc�stica.
 \item Da un ejemplo de matriz estoc�stica y sim�trica.
 \end{enumerate}
 
 \end{block}
 
\end{frame}



\begin{frame}
  \frametitle{Ejercicios de Matrices}
  \begin{block}{Ejercicio 16}
Consideremos la matriz
\[
A= \left(\begin{array}{rrrr}
0&a&a^2&a^3\\
0&0&a&a^2\\
0&0&0&a\\
0&0&0&0
\end{array}\right)\]
A partir de ella definimos la matriz $B$ como
\[B=A-\frac{1}{2}A^2+\frac{1}{3}A^3-\frac{1}{4}A^4+\cdots\]
\[\cdots \]
 \end{block}
 
\end{frame}


\begin{frame}
  \frametitle{Ejercicios de Matrices}
  \begin{block}{Ejercicio 16 (cont.)}
\[\cdots \]

Demostrad que en este sumatorio solamente hay un n�mero finito de t�rminos no nulos y calculad $B$. Demostrad tambi�n que el sumatorio
\[B+\frac{1}{2!}B^2+\frac{1}{3!}B^3+\frac{1}{4!}B^4+\cdots\]
solamente tiene un n�mero finito de t�rminos no nulos, y que su suma vale $A$.

 \end{block}
 
\end{frame}


\end{document}