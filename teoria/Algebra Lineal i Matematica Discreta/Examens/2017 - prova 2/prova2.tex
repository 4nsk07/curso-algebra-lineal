\documentclass[12pt]{article}
\usepackage[catalan]{babel}
\usepackage{amsfonts,amssymb,amsmath,amsthm,hyperref,enumerate, cancel}
\usepackage[utf8]{inputenc}
\usepackage[T1]{fontenc}    
    
\advance\hoffset by -0.8in
\advance\textwidth by 1.6in
\advance\voffset by -0.9in
\advance\textheight by 1.8in
\parskip= 1 ex
\parindent = 10pt
\baselineskip= 13pt



\newcommand{\ZZ}{\mathbb{Z}}
\newcommand{\RR}{\mathbb{R}}
\newcommand{\NN}{\mathbb{N}}
\newcommand{\QQ}{\mathbb{Q}}
\renewcommand{\leq}{\leqslant}
\renewcommand{\geq}{\geqslant}

\newcounter{problemes}
\setcounter{problemes}{0}
\newcounter{punts}
\renewcommand{\thepunts}{\arabic{punts}}
\newcommand{\probl}{\addtocounter{problemes}{1}
\setcounter{punts}{0}
\medskip\noindent{{\bf \theproblemes) }}}
\newcommand{\problm}{\addtocounter{problemes}{1}
\setcounter{punts}{0}
\medskip\noindent{{\bf \theproblemes*) }}}

\newcommand{\punt}{\addtocounter{punts}{1}
\smallskip{{\emph{\thepunts) }}}}
\newcommand{\puntm}{\addtocounter{punts}{1}
\smallskip{{\emph{\thepunts*) }}}}

\newcount\problemes
\problemes=0

\def\probl{\advance\problemes by 1
\vskip 2ex\noindent{\bf \the\problemes \hbox{ } }}


\newcommand{\notimplies}{%
  \mathrel{{\ooalign{\hidewidth$\not\phantom{=}$\hidewidth\cr$\implies$}}}}





\begin{document}
\pagestyle{empty}

\parindent =0 pt
{\bf Algebra Lineal. Primer de Telemàtica. 
\hfill Segon parcial - 22 de Desembre, 2017}
 

\vspace{0.6 cm}
\probl (3.5p) Considerau la següent matriu $A$ donada en funció de dos paràmetres $a,b\in\mathbb{R}$

\[A= \left(\begin{array}{ccc}2a-b & 0 & 2a-2b \\1 & a & 2 \\-a+b & 0 & -a+2b\end{array}\right)\]

\begin{enumerate}
\item (1p) Trobau els valors propis de la matriu $A$ en funció de $a,b$ i digau la seva dimensió algebraica.
\item (1p) En el cas en que $a=b$, trobau els vectors propis associats als valors propis de la matriu $A$ i digau la seva dimensió geomètrica.
\item (1p) En el cas en que $a\neq b$, trobau els vectors propis associats als valors propis de la matriu $A$ i digau la seva dimensió geomètrica.
\item (0.5p) En funció dels resultats anteriors, digau per a quins valors de $a$ i $b$ la matriu $A$ diagonalitza. Trobau en aquests casos la matriu diagonal semblant a la matriu $A$ així com la matriu de canvi de base.
\end{enumerate}
%%Ejercicio 619 libro algebra lineal



\vspace{0.4cm}
\probl (4p) Considerau $\RR_2[x] = \{ax^2+bx+c: a,b,c\in\RR\}$ l'espai de polinomis de grau menor o igual que 2 amb una variable i coeficients reals i considerau l'aplicació lineal 
\[f:\RR_2[x]\longrightarrow \RR\]
que donats un polinomi $p(x)$ l'envia a $f(p(x)) = p(2)$
\begin{enumerate}
\item (0.5p) Demostrau que l'aplicació $f$ és una aplicació lineal.
\item (0.5p) Calculau la matriu de $f$ en la base usual de $\RR_2[x]$, és a dir $B = \{1,x,x^2\}$.
\item (0.5p) Demostrau que els polinomis $q_0(x) = x, q_1(x) = x-1, q_2(x) = x(x-1)$ formen una base de  $\RR_2[x]$, diguem-li $V$.
\item (0.5p) Trobau les matrius de canvi de base de $P_{B\rightarrow V}$ i $P_{V\rightarrow P}$ de l'espai vectorial $\RR_2[x]$.
\item (0.5p) Dibuixau un diagrama que relacioni l'aplicació $f$ respecte de les bases  $B$ i $V$ i la canònica de $\RR$ i calculau la matriu de $f$ en la base $V$ de $\RR_2[x]$.
\item (1.0p) Trobau una base del nucli i una de la imatge de $f$ i discutiu quin tipus d'aplicació és (i.e. monomorfisme, epimorfisme, isomorfisme, automorfisme,...)
\item (0.5p) Demostrau o refutau que el conjunt de polinomis $p\in\RR_2[x]$ tals que $f(p') $ formen un subespai vectorial.
\end{enumerate} 

%Exercicio 721 libro algebra lineal

\vspace{0.6 cm}
\end{document}  