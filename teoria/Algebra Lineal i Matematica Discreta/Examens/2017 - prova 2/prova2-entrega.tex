\documentclass[12pt]{article}
\usepackage[catalan]{babel}
\usepackage{amsfonts,amssymb,amsmath,amsthm,hyperref,enumerate, cancel}
\usepackage[utf8]{inputenc}
\usepackage[T1]{fontenc}    
    
\advance\hoffset by -0.8in
\advance\textwidth by 1.6in
\advance\voffset by -0.9in
\advance\textheight by 1.8in
\parskip= 1 ex
\parindent = 10pt
\baselineskip= 13pt



\newcommand{\ZZ}{\mathbb{Z}}
\newcommand{\RR}{\mathbb{R}}
\newcommand{\NN}{\mathbb{N}}
\newcommand{\QQ}{\mathbb{Q}}
\renewcommand{\leq}{\leqslant}
\renewcommand{\geq}{\geqslant}

\newcounter{problemes}
\setcounter{problemes}{0}
\newcounter{punts}
\renewcommand{\thepunts}{\arabic{punts}}
\newcommand{\probl}{\addtocounter{problemes}{1}
\setcounter{punts}{0}
\medskip\noindent{{\bf \theproblemes) }}}
\newcommand{\problm}{\addtocounter{problemes}{1}
\setcounter{punts}{0}
\medskip\noindent{{\bf \theproblemes*) }}}

\newcommand{\punt}{\addtocounter{punts}{1}
\smallskip{{\emph{\thepunts) }}}}
\newcommand{\puntm}{\addtocounter{punts}{1}
\smallskip{{\emph{\thepunts*) }}}}


\pagestyle{empty}



\begin{document}
\pagestyle{empty}

\parindent =0 pt
{\bf Algebra Lineal. Primer de Telemàtica. 
\hfill Segon parcial - 22 de Desembre, 2017}
 

\vspace{0.6 cm}
\probl (2.5p) Una alimentació sana i equilibrada passa per ingerir aliments que tinguin proteïnes, hidrats de carboni i greixos. 

Anem al mercat a comprar a Desembarc del Rei amb la nostra senalleta. Els quatre mercaders de la zona ens ofereixen diferents delicies de la regió: senglar del nord a l'estaca a 4 venats de plata, serp dorniense cruixent amb mostassa i mel a 2 venats de plata, vi calent de les terres del Rejo a 3 venats de plata el litre i panellets de llimona a l'estil de Sansa a 6 venats de plata el panellet.

Cadascún d'aquests aliments té els següents valors calòrics i alimenticis:


$$
\begin{array}{|c|c|c|c|c|}\hline  & kcal & Prote\text{ï}nes & Hidrats & Greixos \\\hline Senglar\ del\ nord & 300 & 2 & 1 & 1 \\\hline Serp\ dorniense & 200 & 1 & 1 & 2 \\\hline Vi\ calent & 100 & 0 & 2 & 1 \\\hline Panellets\ de\ llimona & 400 & 0 & 3 & 3 \\\hline \end{array}
$$


Cada dia, els guerrers de més enllà del Mur, necessiten ingerir almenys 600 kcal, 20 unitats de proteïnes, 30 d'hidrats i 25 de greix per sobreviure a l'hivern que s'acosta i lluitar contra els caminants blancs. 

Amb tota aquesta informació:
\begin{enumerate}
\item (0.3p) Plantejau el problema de programació lineal pertinent que minimitzi el cost de comprar aliments pel nostre regiment de 500 soldats. 
\item (0.3p) Expressau-lo en forma estàndard 
\item (0.7p) Resoleu-lo pel mètode del símplex i digueu quants aliments de cada hem de comprar per a cada soldat, i quants de venats de plata necessitarem per abastir tot el nostre regiment. 
\item (0.3p) Determinau si la solució actual segueix essent l'òptima en el cas en que el preu unitari del senglar del nord a l'estaca augmenti en un 25$\%$ el seu preu original en venats de plata, i que els panellets de llimona es trobin a un venat de plata de descompte. En cas de que la nova solució no sigui essent l'òptima, trobau la nova solució òptima del problema. 
\item (0.6p) Fins quin preu pot baixar el senglar del nord a l'estaca de manera que la base actual sigui essent l'òptima del problema?
\item (0.3p) Fins quin preu pot baixar el preu dels panellets de llimona de manera que la base actual sigui essent l'òptima del problema?
\end{enumerate}


\textbf{Heu d'entregar aquest exercici de forma individual, com a tard el día del Segon Parcial (22 de Desembre de 2017). Qualsevol sospita de que dues persones s'han copiat o empren un mateix argument per fer l'exercici suposarà tenir un zero directament del mateix.}

\vspace{0.6 cm}
\end{document}  