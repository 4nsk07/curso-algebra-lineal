\documentclass[12p,spanish]{article}
\usepackage[spanish]{babel}
\usepackage[ansinew]{inputenc}
\usepackage[T1]{fontenc}
\usepackage{graphicx}
\usepackage{multicol}
\usepackage{longtable}
\usepackage{array}
\usepackage{multirow}
\usepackage{geometry}                		
\geometry{letterpaper}                   		
\usepackage{graphicx}
\usepackage{amssymb}
\usepackage{color}


\setlength{\textwidth}{16cm}
\setlength{\textheight}{24cm}
\setlength{\oddsidemargin}{-0.3cm}
\setlength{\topmargin}{-1.3cm}


\newcommand{\sC}{{\cal C}}
\newcommand{\sF}{{\cal F}}
\newcommand{\sL}{{\cal L}}
\newcommand{\sU}{{\cal U}}
\newcommand{\sX}{{\cal X}}
\newcommand{\eop}{{\Box}}

\newcommand{\ar}{A^{(r)}}
\newcommand{\HH}{{\bf H}}
\newcommand{\sS}{{\cal S}}
\newcommand{\Img}{\mbox{Img}}

\def\N{I\!\!N}
\def\R{I\!\!R}
\def\Z{Z\!\!\!Z}
\def\Q{O\!\!\!\!Q}
\def\C{C\!\!\!\!I}


\newcount\problemes
\problemes=0

\def\probl{\advance\problemes by 1
\vskip 2ex\noindent{\bf \the\problemes \hbox{ } }}

\graphicspath{ {im/} }


\newcommand{\notimplies}{%
  \mathrel{{\ooalign{\hidewidth$\not\phantom{=}$\hidewidth\cr$\implies$}}}}





\begin{document}
\pagestyle{empty}

\parindent =0 pt
{\bf Algebra Lineal. Primer de Telem�tica. 
\hfill Primer parcial - 11 de Novembre, 2016}
 

\vspace{0.6 cm}
\probl Sigui $A$ la matriu seg�ent
\[A = \left(\begin{array}{ccc}a & 0 & 0 \\1 & a & 0 \\0 & 1 & a\end{array}\right)\]
on $a$ �s un valor real.
\begin{enumerate}
\item (0.5p) Enunciau la condici� necess�ria i suficient per tal que una matriu $A$ tingui inversa. 
\item (1.0p) Calculau la inversa de $A$ per els valors pels quals sigui possible. 
\item (0.5p) Calculau $A^2, A^3$ i $A^4$.
\item (1.0p) Donau una f�rmula general per a l'expressi� de $A^n$.
\end{enumerate}
\vspace{0.6 cm}
\probl (1.5 p) Calculau les arrels de l'equaci�:
\[\left|\begin{array}{cccc}1+x & 1 & \cdots & 1 \\1 & 1+x & \cdots & 1\\\cdots & \cdots & \cdots & \cdots\\1 & 1 & \cdots & 1+x\end{array}\right| = 0\]

\vspace{0.6 cm}
\probl Siguin $\vec u$ i $\vec v$ dos vectors de m�dul 2 i que formen tots dos un �ngle de 60�. 
\begin{enumerate}
\item (1.0p) Quin �s el m�dul de $\vec u + \vec v$? I el de  $\vec u - \vec v$
\item (0.5p) Demostrau que $\vec u + \vec v$ i  $\vec u - \vec v$ s�n perpendiculars.
\end{enumerate}



\vspace{0.6 cm}
\probl Siguin $U=\{u_1,u_2,u_3\}$ i $V=\{v_1,v_2,v_3\}$  dues bases de l'espai vectorial $\mathbb R^3$ relacionades a trav�s del sistema
\[
\left\{\begin{array}{ccl}u_1 & = & v_1-3v_2+4v_3 \\u_2 & = & v_2+v_3 \\u_3 & = & v_1+v_2+v_3\end{array}\right.
\]
\begin{enumerate}
\item (0.5p) Trobau les coordenades dels vector de $U$ en la base $V$. Indicau tamb� la matriu de canvi de base de $U$ a $V$.
\item (1.0p) Trobau les coordenades dels vector de $V$ en la base $U$. Indicau tamb� la matriu de canvi de base de $V$ a $U$.
\item (2.0p) Si $W=\{w_1,w_2,w_3\}$ �s una nova base de  $\mathbb R^3$ amb coordenades respecte de $V$ donades per $w_1=(1,-1,1)_V, w_2=(-1,1,0)_V, w_3 = (0,1,-1)_V$, trobau la matriu de canvi de base de $V$ a $W$ i de $U$ a $W$.
\item (0.5p) Si $x=(1,-1,3)_U$, troba les seves coordenades en la base $W$. 
\end{enumerate}
\vspace{0.6 cm}

\textbf{Temps m�xim per fer la prova: 3 hores. }

\vspace{0.6 cm}
\end{document}  