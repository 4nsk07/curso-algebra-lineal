\documentclass[14p,spanish]{article}
\usepackage[spanish]{babel}
\usepackage[ansinew]{inputenc}
\usepackage[T1]{fontenc}
\usepackage{graphicx}
\usepackage{multicol}
\usepackage{longtable}
\usepackage{array}
\usepackage{multirow}
\usepackage{geometry}                		
\geometry{letterpaper}                   		
\usepackage{graphicx}
\usepackage{amssymb}
\usepackage{color}


\setlength{\textwidth}{16cm}
\setlength{\textheight}{24cm}
\setlength{\oddsidemargin}{-0.3cm}
\setlength{\topmargin}{-1.3cm}


\newcommand{\sC}{{\cal C}}
\newcommand{\sF}{{\cal F}}
\newcommand{\sL}{{\cal L}}
\newcommand{\sU}{{\cal U}}
\newcommand{\sX}{{\cal X}}
\newcommand{\eop}{{\Box}}

\newcommand{\ar}{A^{(r)}}
\newcommand{\HH}{{\bf H}}
\newcommand{\sS}{{\cal S}}
\newcommand{\Img}{\mbox{Img}}

\def\N{I\!\!N}
\def\R{I\!\!R}
\def\Z{Z\!\!\!Z}
\def\Q{O\!\!\!\!Q}
\def\C{C\!\!\!\!I}


\newcount\problemes
\problemes=0

\def\probl{\advance\problemes by 1
\vskip 2ex\noindent{\bf \the\problemes \hbox{ } }}

\graphicspath{ {im/} }


\newcommand{\notimplies}{%
  \mathrel{{\ooalign{\hidewidth$\not\phantom{=}$\hidewidth\cr$\implies$}}}}





\begin{document}
\pagestyle{empty}

\parindent =0 pt
{\bf Algebra Lineal. Primer de Telem�tica. 
\hfill Segon parcial - 19 de Desembre, 2015}
 

\vspace{0.6 cm}
\probl (4p) Al planeta Naboo hi habiten tres especies inteligens, els Gungan d'orelles largues capitanejats per Jar Jar Binks, els Ewook baixets que viuen en grans comunitats i els Wookies peluts capitanejats per Chewbacca. 
\begin{itemize}
\item Sabem que els Gungan i els Wookies es reprodueixen molt r�pidament, de fet cada individu de la seva esp�cie t� 6 nous progenitors d'una generaci� a la seg�ent, mentre que els Wookies nom�s en solen tenir 5. 
\item D'altra banda, les tres esp�cies s�n molt combatives desde que Darth Vader va comen�ar la guerra imperial pel territori del planeta Naboo i es troben en una guerra sense final, ja que es maten entre ells de forma sim�trica: cada Gungan mata dos Ewooks i un Wookie durant la seva vida (i viceversa), mentres que els Ewooks maten als Wookies en relaci� 1:1. 
\item La primera generaci� desp�cies del planeta va comen�ar amb 2 individus de cada esp�cie. 
\end{itemize}
Anem a veure com han evolucionat les esp�cies desde la col�lonitzaci� inicial de Naboo. 
\begin{enumerate}
\item (1p) Escriviu un sistema d'equacions lineals, en forma matricial que ens determini la poblaci� a la generaci� $n$ desde la poblaci� del planeta Naboo. 
\item (1p) Cercau els valors i vectors propis de la matriu de coeficients del problema.
\item (1p) Trobau la seva forma diagonal, i les dues matrius de canvi de base que ens ajudin a calcular quants individus de cada esp�cie t� el planeta en una generaci� donada.
\item (1p) Calculau el nombre d'individus en la generaci� $n$-�ssima. 
\end{enumerate}
%%RESULETO EN REL 6



\vspace{0.4cm}
\probl (4p) Sigui $f:\mathbb R^3\longrightarrow \mathbb R^4$ definida per $f(x,y,z) = (3y-2z, -x+2y+2z, y+z,x+y-2z)$ respecte de les bases $B_{\mathbb R^3} =\{(1,0,1), (1,1,1), (1,0,0)\}$ i $B_{\mathbb R^4}=\{(-2,1,1,-1), (1,3,2,0), (0,-1,0,1), (0,0,0,1)\}$.
\begin{enumerate}
\item (0.75p) Demostrau que l'aplicaci� $f$ �s una aplicaci� lineal. 
\item (0.75p) Trobau el nucli i la imatge de $f$. Discutiu quin tipus d'aplicaci� �s (i.e. monomorfisme, epimorfisme, isomorfisme, automorfisme,...)
\item (0.25p) Calculau la matriu de $f$ respecte de les bases $B_{\mathbb R^3}$ i $B_{\mathbb R^4}$.
\item (0.25p) Dibuixau un diagrama que relacioni l'aplicaci� $f$ respecte de les bases  $B_{\mathbb R^3}$ i $B_{\mathbb R^4}$ amb la seva matriu equivalent respecte de les bases can�niques respectives d'ambdos espais vectorials.
\item (2.0p) Calculau la matriu de $f$ respecte de les bases can�niques respectives. 
\end{enumerate} 

%\textbf{Heu d'entregar l'exercici  com a tard el d�a del Segon Parcial (19 de Desembre de 2015). }

\vspace{0.6 cm}
\end{document}  