\documentclass[12p,spanish]{article}
\usepackage[spanish]{babel}
\usepackage[ansinew]{inputenc}
\usepackage[T1]{fontenc}
\usepackage{graphicx}
\usepackage{multicol}
\usepackage{longtable}
\usepackage{array}
\usepackage{multirow}
\usepackage{geometry}                		
\geometry{letterpaper}                   		
\usepackage{graphicx}
\usepackage{amssymb}
\usepackage{color}


\setlength{\textwidth}{16cm}
\setlength{\textheight}{24cm}
\setlength{\oddsidemargin}{-0.3cm}
\setlength{\topmargin}{-1.3cm}


\newcommand{\sC}{{\cal C}}
\newcommand{\sF}{{\cal F}}
\newcommand{\sL}{{\cal L}}
\newcommand{\sU}{{\cal U}}
\newcommand{\sX}{{\cal X}}
\newcommand{\eop}{{\Box}}

\newcommand{\ar}{A^{(r)}}
\newcommand{\HH}{{\bf H}}
\newcommand{\sS}{{\cal S}}
\newcommand{\Img}{\mbox{Img}}

\def\N{I\!\!N}
\def\R{I\!\!R}
\def\Z{Z\!\!\!Z}
\def\Q{O\!\!\!\!Q}
\def\C{C\!\!\!\!I}


\newcount\problemes
\problemes=0

\def\probl{\advance\problemes by 1
\vskip 2ex\noindent{\bf \the\problemes \hbox{ } }}

\graphicspath{ {im/} }


\newcommand{\notimplies}{%
  \mathrel{{\ooalign{\hidewidth$\not\phantom{=}$\hidewidth\cr$\implies$}}}}





\begin{document}
\pagestyle{empty}

\parindent =0 pt
{\bf Problemes d'Algebra Lineal. Primer de Telem�tica. 
\hfill Tema 4 - Aplicacions Lineals}

\vspace{0.6 cm}
\probl Estudiau les seg�ents aplicacions i comprovau que s�n lineals. Calculau-ne el nucli, la imatge i la dimensi�: 
\begin{multicols}{2}
\begin{enumerate}
\item   \[
  \begin{array}{@{}r@{\;}c@{\;}c@{\;}l@{}}
    f: & \mathbb R^2 & \rightarrow & \mathbb R,   \\
       & (x, y) & \mapsto     & y.
  \end{array}
\]
\item   \[
  \begin{array}{@{}r@{\;}c@{\;}c@{\;}l@{}}
    f: & \mathbb R^2 & \rightarrow & \mathbb R,   \\
       & (x, y) & \mapsto     & x+y.
  \end{array}
\]
\item   \[
  \begin{array}{@{}r@{\;}c@{\;}c@{\;}l@{}}
    f: & \mathbb R^3 & \rightarrow & \mathbb R,   \\
       & (x, y,z) & \mapsto     & x+y.
  \end{array}
\]
\item   \[
  \begin{array}{@{}r@{\;}c@{\;}c@{\;}l@{}}
    f: & \mathbb R^3 & \rightarrow & \mathbb R,   \\
       & (x, y,z) & \mapsto     & x-y+z.
  \end{array}
\]
\item   \[
  \begin{array}{@{}r@{\;}c@{\;}c@{\;}l@{}}
    f: & \mathbb R^3 & \rightarrow & \mathbb R^2,   \\
       & (x, y,z) & \mapsto     & (x,y).
  \end{array}
\]

\item   \[
  \begin{array}{@{}r@{\;}c@{\;}c@{\;}l@{}}
    f: & \mathbb R^3 & \rightarrow & \mathbb R^2,   \\
       & (x, y,z) & \mapsto     & (x+y,z).
  \end{array}
\]
\item   \[
  \begin{array}{@{}r@{\;}c@{\;}c@{\;}l@{}}
    f: & \mathbb R^3 & \rightarrow & \mathbb R^2,   \\
       & (x, y,z) & \mapsto     & (x-y,y+z).
  \end{array}
\]
\item   \[
  \begin{array}{@{}r@{\;}c@{\;}c@{\;}l@{}}
    f: & \mathbb R^3 & \rightarrow & \mathbb R^2,   \\
       & (x, y,z) & \mapsto     & (x+y-2z,0).
  \end{array}
\]
\item   \[
  \begin{array}{@{}r@{\;}c@{\;}c@{\;}l@{}}
    f: & \mathbb R^3 & \rightarrow & \mathbb R^2,   \\
       & (x, y,z) & \mapsto     & (x+y-5,y-z).
  \end{array}
\]
\item   \[
  \begin{array}{@{}r@{\;}c@{\;}c@{\;}l@{}}
    f: & \mathbb R^3 & \rightarrow & \mathbb R^2,   \\
       & (x, y,z) & \mapsto     & (2x,1).
  \end{array}
\]
\end{enumerate}
\end{multicols}


\vspace{0.4cm}
\probl Obteniu els subespais vectorials nucli i imatge de les seg�ents aplicacions lineals:
\begin{multicols}{2}
\begin{enumerate}
\item   \[
  \begin{array}{@{}r@{\;}c@{\;}c@{\;}l@{}}
    f: & \mathbb R^2 & \rightarrow & \mathbb R^2,   \\
       & (x, y) & \mapsto     & (2x-y, x+y).
  \end{array}
\]
\item   \[
  \begin{array}{@{}r@{\;}c@{\;}c@{\;}l@{}}
    f: & \mathbb R^3 & \rightarrow & \mathbb R^3,   \\
       & (x, y,z) & \mapsto     & (x+2y+z, x+5y, z).
  \end{array}
\]
\item   \[
  \begin{array}{@{}r@{\;}c@{\;}c@{\;}l@{}}
    f: & \mathbb R^2 & \rightarrow & \mathbb R^4,   \\
       & (x, y) & \mapsto     & (x, -y, x+3y, x-y).
  \end{array}
\]
\item   \[
  \begin{array}{@{}r@{\;}c@{\;}c@{\;}l@{}}
    f: & \mathbb R^4 & \rightarrow & \mathbb R^3,   \\
       & (x, y,z,t) & \mapsto     & (7x+2y-z+t, y+z, -x).
  \end{array}
\]
\end{enumerate}
\end{multicols}

\vspace{0.4cm}
\probl Sigui $f:\mathbb R^3\longrightarrow \mathbb R^2$ definida per $f(e_1)=(1,1), f(e_2) = (3,0), f(e_3)=(4,7)$ on $\{e_1, e_2, e_3\}$ �s la base can�nica de $\mathbb R^3$.
\begin{enumerate}
\item Calcular $f(1,3,8)$ i $f(x,y,z)$
\item Determinar $Ker(f)$ i $Im(f)$
\end{enumerate}

\vspace{0.4cm}
\probl Sigui $f:\mathbb R^3\longrightarrow \mathbb R^2$ definida per $f(x,y,z) = (3x+2y-4z, x-5y+3z)$ 
\begin{enumerate}
\item Calculau la matriu de $f$ respecte de les bases can�niques.
\item Calculau la matriu de $f$ respecte de les bases $B_{\mathbb R^3} =\{(1,1,1),(1,1,0),(1,0,0)\}$ i $B_{\mathbb R^2}=\{(1,1), (0,1)\}$.
\end{enumerate} 


\vspace{0.4cm}
\probl Sigui $f:\mathbb R^3\longrightarrow \mathbb R^3$ amb matriu associada respecte de la base can�nica $A$ donada per
\[A=\left(\begin{array}{ccc}3 & -1 & 1 \\0 & 2 & 0 \\1 & -1 & 3\end{array}\right)\] 
Determinar la matriu $C$ associada a $f$ respecte de la base 
\[B'=\{(1,0,-1), (0,1,1), (1,0,1)\}\]

\vspace{0.4cm}
\probl Sigui $f$ l'endomorfisme de $\mathbb R^3$ definit per $f(e_1) = -e_1, f(e_2) = e_1+e_2+e_3, f(e_3) = -e_2-e_3$ amb $\{e_1,e_2,e_3\}$ la base can�nica de $\mathbb R^3$
\begin{enumerate}
\item Determinau la matriu $f$ respecte de la base can�nica anterior.
\item Trobau la dimensi� del nucli i la imatge de $f$
\item Provau que els vectors $u_1=-e_2, u_2=e_1+e_3, u_3 = e_1$ formen una base de $\mathbb R^3$ i trobau la matriu de $f$ respecte d'aquesta base.
\end{enumerate}

\vspace{0.4cm}
\probl Un endomorfisme $f$ de $\mathbb R^3$ est� determinat per $f(x,y,z)=(2y+z, x-4y, 3x)$ en la base can�nica. Se demana:
\begin{enumerate}
\item Nucli i imatge de $f$
\item La matriu de $f$ en aquesta base
\item La matriu de $f$ en la base constituida pels vectors $v_1=(1,1,1), v_2=(1,1,0), v_3=(1,0,0)$
\item L'expressi� de $f$ en la base $V$.
\end{enumerate}

\vspace{0.4cm}
\probl Determinau la matriu del morfisme $f:\mathbb R^3 \longrightarrow \mathbb R^4$ tal que $f(0,1,1) = (1,2,7,1), f(1,0,3) = (-1,2,3,1), f(2,-1,0) = (2,0,4,0)$. Trobau les bases de $Ker(f)$. Quines s�n les anti-imatges del vector $(2,4,14,2)$?


\end{document}  