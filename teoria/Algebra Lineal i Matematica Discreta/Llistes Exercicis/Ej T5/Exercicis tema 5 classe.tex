\documentclass[12p,spanish]{article}
\usepackage[spanish]{babel}
\usepackage[ansinew]{inputenc}
\usepackage[T1]{fontenc}
\usepackage{graphicx}
\usepackage{multicol}
\usepackage{longtable}
\usepackage{array}
\usepackage{multirow}
\usepackage{geometry}                		
\geometry{letterpaper}                   		
\usepackage{graphicx}
\usepackage{amssymb}
\usepackage{color}


\setlength{\textwidth}{16cm}
\setlength{\textheight}{24cm}
\setlength{\oddsidemargin}{-0.3cm}
\setlength{\topmargin}{-1.3cm}


\newcommand{\sC}{{\cal C}}
\newcommand{\sF}{{\cal F}}
\newcommand{\sL}{{\cal L}}
\newcommand{\sU}{{\cal U}}
\newcommand{\sX}{{\cal X}}
\newcommand{\eop}{{\Box}}

\newcommand{\ar}{A^{(r)}}
\newcommand{\HH}{{\bf H}}
\newcommand{\sS}{{\cal S}}
\newcommand{\Img}{\mbox{Img}}

\def\N{I\!\!N}
\def\R{I\!\!R}
\def\Z{Z\!\!\!Z}
\def\Q{O\!\!\!\!Q}
\def\C{C\!\!\!\!I}


\newcount\problemes
\problemes=0

\def\probl{\advance\problemes by 1
\vskip 2ex\noindent{\bf \the\problemes \hbox{ } }}

\graphicspath{ {im/} }


\newcommand{\notimplies}{%
  \mathrel{{\ooalign{\hidewidth$\not\phantom{=}$\hidewidth\cr$\implies$}}}}





\begin{document}
\pagestyle{empty}

\parindent =0 pt
{\bf Problemes de Classe d'Algebra Lineal. Primer de Telem�tica. 
\hfill Tema 5 - Diagonalitzaci�}

\vspace{0.6 cm}
\probl Donada la matriu
\[A = \left(\begin{array}{ccc}-5 & -5 & -9  \\8 & 9 & 18 \\ -2 & -3 & -7\end{array}\right)\]
Comprovau que $\lambda = -1$ �s un valor propi de $A$ i que $(3,-6,2)$ �s un vector propi associat a $\lambda$. �s el vector $(1,2,-1)$ un vector propi associat a $\lambda = -1$?

\vspace{0.4 cm}
\probl Justificau si s�n diagonalitzables les seg�ents matrius:
\[A = \left(\begin{array}{ccc}2 & 1 & 0  \\0 & 1 & -1 \\ 0 & 2 & 4\end{array}\right)\]
\[B = \left(\begin{array}{ccc}-1 & 0 & 0  \\0 & 0 & -1 \\ 0 & 1 & 0\end{array}\right)\]

\vspace{0.4 cm}
\probl Provau si s�n o no diagonalitzables les seg�ents matrius de $M_3(\mathbb R)$ i, en cas de que  ho siguin, trobau una matriu $P$ de vectors propis i la matriu diagonal.
\[A = \left(\begin{array}{ccc}a & 1 & 1  \\1 & a & 0 \\ 0 & 0 & a\end{array}\right)\]
\[B = \left(\begin{array}{ccc}a & 0 & a  \\1 & a+1 & -2 \\ -1 & -1 & 2\end{array}\right)\]
\[C = \left(\begin{array}{ccc}a & 1 & a-1  \\1 & 2a & -1 \\ 2a+1 & 1 & -2\end{array}\right)\]
\[D = \left(\begin{array}{ccc}a+1 & a+b & b  \\-a & -a & -1 \\ a & a-1 & 0\end{array}\right)\]

\vspace{0.4 cm}
\probl Sigui $A$ una matriu real, quadrada d'ordre $p$ amb tots els coeficients igual a $1$.
\begin{enumerate}
\item Demostrau que $A^n = p^{n-1}A$ per a tot enter $n\geq 1$.
\item Calculau els valors propis de $A$.
\item Trobau, si es possible,  una matriu $P$ tal que $P^-1AP$ sigui diagonal i calculau $P^-1$.
\end{enumerate}

\vspace{0.4 cm}
\probl Donada la matriu
\[A = \left(\begin{array}{ccc}-1 & 0 & -3  \\3 & 2 & 3 \\ -3 & 0 & -1\end{array}\right)\]
\begin{enumerate}
\item Provau que $A$ �s diagonalitzable
\item Calculau $A^n$ per a tot $n\geq 1$
\item Provau que $p_A(A)=0$ on $p_A(x)$ �s el polinomi caracter�stic de la matriu $A$.
\end{enumerate}


\vspace{0.4 cm}
\probl Donades les matrius
\[A = \left(\begin{array}{ccc}1 & 1 & 0  \\-1 & 0 & 0 \\ 2 & 0 & -1\end{array}\right)\]
\[A = \left(\begin{array}{ccc}0 & 0 & 1  \\1 & 0 & -2 \\ 0 & 1 & 1\end{array}\right)\]
Calculau, utilitzant el teorema de Cayley-Hamilton $A^{-1}, B^4$ i $B^5$

\vspace{0.4 cm}
\probl Donat $a\in \mathbb R$, considerau la matriu
\[A = \left(\begin{array}{ccc}0 & 0 & 1  \\1 & 0 & -1 \\ -1 & 1 & a\end{array}\right)\]
\begin{enumerate}
\item Demostrau que $A^3-aA^2+2A-I_3=0$
\item Demostrau que $A$ �s invertible i calculau $A^{-1}$
\item Trobau el valor de $A^5-aA^4+A^3-(1-a)A^2-a+I_3$
\end{enumerate}

\vspace{0.4 cm}
\probl Sigui $A\in M_n(\mathbb R) $ una matriu quadrada d'ordre $n$. Demostrau que els valors propis de $A$ i de $A^t$ coincideixen.

\vspace{0.4 cm}
\probl Calculau $A^n$ per a tot $n\in \mathbb N$, si la matriu $A$ �s
\[A = \left(\begin{array}{ccc}1 & 1 & 1  \\1 & -2 & 0 \\ 0 & 3 & 1\end{array}\right)\]

\vspace{0.4 cm}
\probl Trobau el terme general de la successi� $a_n/b_n$ definida per
\[a_1=b_1 = 1,\ \ \ \ a_{n+1} = a_n+2b_n,\ \ \ \ b_{n+1} = a_n+b_n\]

\vspace{0.4 cm}
\probl Considereu les successions definides recurrentment per a tot $n\geq 1$ per 
\[u_n = -4u_{n-1}-6v_{n-1}\ \ \ \ v_{n} = 3u_{n-1}+5v_{n-1} \ \ \ \ w_n = 3u_{n-1}+6v_{n-1}+5w_{n-1}\]
Calculau $u_n, v_n, w_n$ en funci� de $u_0, v_0, w_0$.

\end{document}  