\documentclass[aspectratio=169]{beamer}
\usepackage[spanish]{babel}
\usepackage[latin1]{inputenc}
\usepackage{multicol} % indice en 2 columnas
\usepackage{centernot}
\usepackage{amsmath}% http://ctan.org/pkg/amsmath

\newcommand{\notimplies}{%
  \mathrel{{\ooalign{\hidewidth$\not\phantom{=}$\hidewidth\cr$\implies$}}}}


\usetheme{Warsaw}
%\usecolortheme{crane}
\useoutertheme{shadow}
\useinnertheme{rectangles}
\usecolortheme{orchid}

\setbeamertemplate{navigation symbols}{} % quitar simbolitos




\title[Tema 1 - Matrices, sistemas y determinantes]{C\'{a}lculo matricial}
\subtitle{Estudios de Ingenier\'ia}
\author[https://frogames.es]{
Juan Gabriel Gomila%$^{1}$  \and E. Eva$^{2}$ \and S. Serpiente$^{3}$
}
\institute[Frogames]{
 % $^{1-2}$
 Frogames
   \and
  \texttt{https://frogames.es}
}
\date{\today}


\AtBeginSection{
\begin{frame}
  \begin{multicols}{2}
  \tableofcontents[currentsection]   
\end{multicols}
\end{frame}
}

\AtBeginSubsection{
\begin{frame}
  \begin{multicols}{2}
  \tableofcontents[currentsection,currentsubsection]
\end{multicols}
\end{frame}
}



%empieza aqui


\begin{document} 

\frame{\titlepage}


\section{Matrices}
\subsection{Definiciones generales}


%Definici� de Matriu

\begin{frame}
  \frametitle{?`Qu\'e es una matriz?}
  \begin{block}{Definici\'{o}n de matriz}
Sea $(\mathbb{K}, +, .)$ un cuerpo conmutativo y $m, n \geq 1$ enteros. Una matriz $m \times n$ sobre $\mathbb{K}$ (o de orden $m \times n$ sobre $\mathbb{K}$) es una tabla formada por elementos de $\mathbb{K}$ dispuestos en $m$ filas y $n$ columnas de la forma:
\[\   \left(\begin{matrix} % or pmatrix or bmatrix or Bmatrix or ...
      a_{11} & a_{12} & \cdots & a_{1n} \\
 a_{21} & a_{22} & \cdots & a_{2n} \\
  \vdots & \vdots & \ddots & \vdots \\
   a_{m1} & a_{m2} & \cdots & a_{mn} \\
    \end{matrix}\right)
    \ \mathrm{amb}\ a_{ij} \in \mathbb{K}; i=1,2,...,m; j =1,2,...,n\]

  \end{block}
\end{frame}


\begin{frame}
  \frametitle{?`Qu\'{e} es una matriz?}   
  \begin{block}{Coeficientes de la matriz}
Cada $a_{ij}$ se denomina t\'ermino, coeficiente o entrada de la matriz $A$. El primer sub\'indice, $i$, indica el n\'umero de la fila y el segundo, $j$, el de la columna que ocupa el t\'ermino en la matriz.
  \end{block}
\end{frame}

% Conjunt de matrius

\begin{frame}
  \frametitle{?`D\'onde est\'an las matrices?}
  \begin{block}{Conjunto de matrices}
Se denotar\'a por $M_{m\times n}(\mathbb{K})$ el conjunto de todas las matrices de orden $m\times n$ sobre $\mathbb{K}$. Una matriz cualquiera de $M_{m\times n}(\mathbb{K})$ se denotar\'a indistintamente por $A$, por $(a_{ij})_{m\times n}$ o simplemente por $(a_{ij})$. 
  \end{block}

  \begin{block}{Matrices cuadradas}
 Cuando $m = n$, el conjunto de todas las matrices de orden $M_{n\times n}$ se denota simplemente por $M_{n}(\mathbb{K})$ (las matrices que se clasifican como cuadradas se dicen que son de orden $n$ en vez de $n \times n$ como veremos m\'as adelante).
  \end{block}
\end{frame}



% Igualtat de matrius

\begin{frame}
  \frametitle{?`Cu\'ando son dos matrices iguales?}
  \begin{block}{Igualdad de matrices}
Dadas dos matrices del mismo orden $m \times n$, $A = (a_{ij} )_{m\times n}$ y $B = (b_{ij} )_{m\times n}$ son iguales si:
\[a_{ij} = b_{ij} \forall\  i = 1,...,m, \ \forall\ j = 1,...,n.\]
  \end{block}

 \end{frame}


\subsection{Tipos de matrices}

 %fila
 \begin{frame}
  \frametitle{Tipos de matrices}
  \begin{block}{Matriz fila}
Se denomina matriz fila a toda matriz que consta de una \'unica fila:
\[ A = (a_{11}, a_{12}, \cdots, a_{1n})\in M_{1\times n} (\mathbb{K})\]
  \end{block}

 \end{frame}

%columna
\begin{frame}
  \frametitle{Tipos de matrices}
  \begin{block}{Matriz columna}
Se denomina matriz columna a toda matriz que consta de una \'unica columna:
\[ A = \left(\begin{matrix} % or pmatrix or bmatrix or Bmatrix or ...
      a_{11}  \\
      a_{21} \\
      \vdots \\
      a_{m1}
   \end{matrix}\right) \in M_{m\times 1} (\mathbb{K})\]
  \end{block}

 \end{frame}


%quadrada
 \begin{frame}
  \frametitle{Tiposs de matrices}
  \begin{block}{Matriu cuadrada}
Se denomina matriz cuadrada de orden $n$ a toda matriz que consta de $n$ filas y $n$ columnas
\[ A =  \left(\begin{matrix} % or pmatrix or bmatrix or Bmatrix or ...
      a_{11} & a_{12} & \cdots & a_{1n} \\
 a_{21} & a_{22} & \cdots & a_{2n} \\
  \vdots & \vdots & \ddots & \vdots \\
   a_{n1} & a_{n2} & \cdots & a_{nn} \\
    \end{matrix}\right) \in M_{ n} (\mathbb{K})\]
  \end{block}

  

 \end{frame}
 
 
  \begin{frame}
  \frametitle{Matrices cuadradas}
  
 Dentro del \'ambito de las matrices cuadradas caben las siguientes definiciones y tipos particulares de matrices:


  \begin{block}{Diagonal principal}
Se denomina diagonal (principal) de una matriz cuadrada $A$ a los elementos $a_{ii}$ con $i=1,\cdots, n$.
\[ A =  \left(\begin{matrix} % or pmatrix or bmatrix or Bmatrix or ...
      \boldsymbol{a_{11}} & a_{12} & \cdots & a_{1n} \\
 a_{21} & \boldsymbol{a_{22}} & \cdots & a_{2n} \\
  \vdots & \vdots & \ddots & \vdots \\
   a_{n1} & a_{n2} & \cdots & \boldsymbol{a_{nn}} \\
    \end{matrix}\right) \in M_{ n} (\mathbb{K})\]
  \end{block}

 \end{frame}
 
   \begin{frame}
  \frametitle{Matrices cuadradas}
  

   \begin{block}{Matriz diagonal}
Una matriz diagonal es aquella en la cual $a_{ij} = 0$ siempre que $i\neq j $
\[ A =  \left(\begin{matrix} % or pmatrix or bmatrix or Bmatrix or ...
      a_{11} & 0 & \cdots & 0 \\
      0 & a_{22} & \cdots & 0 \\
  \vdots & \vdots & \ddots & \vdots \\
   0 & 0 & \cdots & a_{nn} \\
    \end{matrix}\right) \in M_{ n} (\mathbb{K})\]
  \end{block}

 \end{frame}
 
 
   \begin{frame}
  \frametitle{Matrices cuadradas}
   \begin{block}{Matriz escalar}
Una matriz escalar es una matriz diagonal en la cual $a_{ii} = \lambda$,  $\forall i = 1, \cdots, n $
\[ A =  \left(\begin{matrix} % or pmatrix or bmatrix or Bmatrix or ...
      \lambda & 0 & \cdots & 0 \\
      0 & \lambda & \cdots & 0 \\
  \vdots & \vdots & \ddots & \vdots \\
   0 & 0 & \cdots & \lambda \\
    \end{matrix}\right) \in M_{ n} (\mathbb{K})\]
  \end{block}
 \end{frame}
 
 



 \begin{frame}
  \frametitle{Matrices cuadradas}
   \begin{block}{Matriz identidad}
Se denomina matriz unidad o matriz identidad de orden $n$, y se denota como $I_n$ a la matriz escalar en la cual todos los elementos de la diagonal son unos. 
\[ I_n =  \left(\begin{matrix} % or pmatrix or bmatrix or Bmatrix or ...
      1 & 0 & \cdots & 0 \\
      0 & 1 & \cdots & 0 \\
  \vdots & \vdots & \ddots & \vdots \\
   0 & 0 & \cdots & 1 \\
    \end{matrix}\right) \in M_{ n} (\mathbb{K})\]
  \end{block}
 \end{frame}
 
 
 
  \begin{frame}
  \frametitle{Matrices cuadradas}
   \begin{block}{Matriz triangular superior}
Se denomina matriz triangular superior a toda matriz en la cual $a_{ij} = 0$,  $\forall i > j $. Es decir, todos los elementos situados por debajo de la diagonal principal son nulos. 
\[ A =  \left(\begin{matrix} % or pmatrix or bmatrix or Bmatrix or ...
      a_{11} & a_{12} & \cdots & a_{1n} \\
      0 & a_{22} & \cdots & a_{2n} \\
  \vdots & \vdots & \ddots & \vdots \\
   0 & 0 & \cdots & a_{nn} \\
    \end{matrix}\right) \in M_{ n} (\mathbb{K})\]
  \end{block}
 \end{frame}


 \begin{frame}
  \frametitle{Matrices cuadradas}
   \begin{block}{Matriz triangular inferior}
Se denomina matriz triangular inferior a toda matriz en la cual  $a_{ij} = 0$,  $\forall i < j $. Es decir, todos los elementos situados por encima de la diagonal principal son nulos.
\[ A =  \left(\begin{matrix} % or pmatrix or bmatrix or Bmatrix or ...
      a_{11} & 0 & \cdots & 0 \\
      a_{21} & a_{22} & \cdots & 0 \\
  \vdots & \vdots & \ddots & \vdots \\
   a_{n1} & a_{n2} & \cdots & a_{nn} \\
    \end{matrix}\right) \in M_{ n} (\mathbb{K})\]
  \end{block}
 \end{frame}
 
 
  \begin{frame}
  \frametitle{Cas general}
  
Para matrices en general (no necesariamente cuadradas) se mantendr\'a la denominaci\'on de matriz triangular superior cuando $a_{ij} = 0\ \forall \ i > j$. M\'as adelante se estudia\'an en profundidad unos tipos especiales de estas matrices (las matrices escalonadas) que tendr\'an una importancia determinante en nuestros estudios.  
 \end{frame}


  \begin{frame}
  \frametitle{Caso general}
  Las matrices triangulares superiores, si no son cuadradas, se corresponden con los siguientes casos dependiendo de si $m < n$ o $n < m$ respectivamente:
\[   \left(\begin{matrix} % or pmatrix or bmatrix or Bmatrix or ...
      a_{11} & a_{12} & \cdots & a_{1m} & \cdots & a_{1n}  \\
      0 & a_{22} & \cdots & a_{2m} & \cdots & a_{2n} \\
  \vdots & \vdots & \ddots & \vdots & &\vdots \\
   0 & 0 & \cdots & a_{mm} & \cdots & a_{mn} \\
    \end{matrix}\right) 
     \left(\begin{matrix} % or pmatrix or bmatrix or Bmatrix or ...
      a_{11} & a_{12} & \cdots & a_{1n}  \\
      0 & a_{22} & \cdots & a_{2n} \\
  \vdots & \vdots & \ddots & \vdots \\
   0 & 0 & \cdots  & a_{mn} \\
   0 & 0 & \cdots  & 0 \\
     \vdots & \vdots & \ddots & \vdots \\
   0 & 0 & \cdots  & 0 \\
    \end{matrix}\right) \]
 \end{frame}


 \begin{frame}
  \frametitle{Matrices cuadradas}
   \begin{block}{Matriz nula}
Se denota como $O$ a la matriu nula, matriz con todos sus coeficientes nulos.\[ A =  \left(\begin{matrix} % or pmatrix or bmatrix or Bmatrix or ...
      0 & 0 & \cdots & 0 \\
      0 & 0 & \cdots & 0 \\
  \vdots & \vdots & \ddots & \vdots \\
   0 & 0 & \cdots & 0 \\
    \end{matrix}\right) \in M_{ n} (\mathbb{K})\]
  \end{block}
 \end{frame}
 
 
 
 
 
 \subsection{Operaciones con matrices}

 \begin{frame}
  \frametitle{Operaciones con matrices}
   \begin{block}{Suma de matrices}
La suma de dos matrices $A$ y $B$ solo es posible si ambas son del mismo orden $m\times n$, entonces se suman t\'ermino a t\'ermino. Es decir, dadas $A=(a_{ij})_{m\times n}$ y $B=(b_{ij})_{m\times n}\in M_{m\times n}(\mathbb{K})$, se define la suma de $A$ y $B$ como la matriz: 
\[C=(c_{ij})_{m\times n}\ \mathrm{on} \ c_{ij} = a_{ij}+b_{ij},\]
\[ \forall \ i=1,\cdots, m, \ \forall\ j=1,\cdots,n\]  
  \end{block}
 \end{frame}


 \begin{frame}
  \frametitle{Operaciones con matrices}
   \begin{block}{Producto por un escalar}
Sea $a \in \mathbb{K}$ y $A = (a_{ij})_{m\times n} \in M_{m\times n}(\mathbb{K})$, se define el producto $aA$ como una nueva matriz de orden $m \times n$ dada por:
\[aA = (a \cdot a_{ij})_{m\times n}\]
  \end{block}
 \end{frame}
 
 
  \begin{frame}
  \frametitle{Operaciones con matrices}
   \begin{block}{Producto de matrices}
Para poder realizar el producto de una matriz $A$ por una matriz $B$, el n\'umero de columnas de $A$ ha de coincidir con el n\'umero de filas de $B$, entonces cada entrada $ij$ de la matriz producto se obtiene multiplicando la fila $i$ de $A$ por la columna $j$ de $B$. 
  \end{block}
 \end{frame}
 
   \begin{frame}
  \frametitle{Operaciones con matrices}
 Concretamente, si $A \in M_{m\times n}(\mathbb{K})$ y $B \in M_{n\times p}(\mathbb{K})$, el producto $AB$ es una matriz $C \in M_{m\times p}(\mathbb{K})$ definida como:

\[  \left( \begin{matrix} % or pmatrix or bmatrix or Bmatrix or ...
      a_{11} & a_{12} & a_{13} & \cdots & a_{1n} \\
      a_{21} & a_{22} & a_{23} & \cdots & a_{2n} \\
      \vdots & \vdots & \vdots &   & \vdots \\
      a_{i1} & a_{i2} & a_{i3} & \cdots & a_{in} \\
      \vdots & \vdots & \vdots &   & \vdots \\
      a_{m1} & a_{m2} & a_{m3} & \cdots & a_{mn} \\
   \end{matrix}
   \right)  
   \left( \begin{matrix} % or pmatrix or bmatrix or Bmatrix or ...
      b_{11} & b_{12} & \cdots & b_{1j} & \cdots & b_{1p} \\
      b_{21} & b_{22} & \cdots & b_{2j} & \cdots & b_{2p} \\
       b_{31} & b_{32} & \cdots & b_{3j} & \cdots & b_{3p} \\
      \vdots & \vdots & & \vdots &   & \vdots \\
      b_{n1} & b_{n2} & \cdots & b_{nj} & \cdots & b_{np} \\
   \end{matrix}
   \right) = (c_{ij})\]
   con $c_{ij} = a_{i1}b_{1j} + a_{i2}b_{2j}+a_{i3}b_{3j}+\cdots+a_{in}b_{nj} = \sum_{k=1}^n a_{ik}b_{kj}$.
   
   N\'otese que $A_{m\times \boldsymbol{n}}\cdot B_{\boldsymbol{n}\times p} = C_{m\times p}$. 
 \end{frame}
 
 
 %propietats
 \subsection{Propiedades}

  \begin{frame}
  \frametitle{Propiedades caracter\'isticas}
  
  Siempre que tengan sentido les operacions indicadas (es decir, las matrices son de los \'ordenes adecuados para poder realizarlas) se satisfacen las siguientes propiedades:
   \begin{block}{Propiedad conmutativa}
   \[A+B = B+A\]
  \end{block}
  
  
     \begin{block}{Propiedad asociativa de la suma}
   \[(A+B)+C= A+(B+C)\]
  \end{block}
  
  
 \end{frame}
 
  
   \begin{frame}
  \frametitle{Propiedades caracter\'isticas}
  
   \begin{block}{Elemento neutro de la suma o elemento nulo}
   \[A+O = O+A = A\]
  \end{block}
  
  
     \begin{block}{Matriz opuesta}
   $\forall \ A = (a_{ij})_{m\times n}$ existe $-A = (-a_{ij})_{m\times n}$ tal que \[A+(-A) = (-A)+A = O\]
  \end{block}
 \end{frame}
 
 
 
   \begin{frame}
  \frametitle{Propiedades caracter\'isticas}
  
   \begin{block}{Propiedad asociativa del producto}
   \[(AB)C = A(BC)\]
  \end{block}
  
  
     \begin{block}{Propiedad distributiva del producto repecto de la suma}
   \[A(B+C) = AB+AC\]
  \end{block}
  
  
 \end{frame}
 
  
   \begin{frame}
  \frametitle{Propiedades caracter\'isticas}
  
   \begin{block}{Elemento neutro del producto o elemento unidad}
   \[AI_n = A \]
   \[I_nB = B\]
  \end{block}
  
  
     \begin{block}{Propiedad distributiva del producto por escalares respecto de la suma}
   \[a(B+C) = aB + aC,\ a\in \mathbb{K} \]
  \end{block}
  
  
 \end{frame}
 
   
   \begin{frame}
  \frametitle{Propiedades caracter\'isticas}
  
   \begin{block}{Elemento neutro del producto por escalar}
   \[1A = A \]
  \end{block}
  
  
     \begin{block}{Propiedad distributiva del producto por matrices respecto de la suma de escalares}
   \[(a+b)C = aC + bC, \ a,b\in\mathbb{K}\]
  \end{block}
  
  
 \end{frame}
 
 
    
   \begin{frame}
  \frametitle{Propiedades caracter\'isticas}
  
   \begin{block}{Propiedad asociativa del producto de escalares por una matriz}
   \[(ab)C = a(bC), \ a,b\in\mathbb{K} \]
  \end{block}
  
  
     \begin{block}{Propiedad asociativa del producto de un escalar por dos matrices}
   \[a(BC) = (aB)C, \ a\in\mathbb{K}\]
  \end{block}
  
  
 \end{frame}
 
 
 
 %%Exemple 1
   \begin{frame}
  \frametitle{Ejemplos}
  
   \begin{block}{Ejemplo}
   \[(AB)C=A(BC) \]
  \end{block}
   \begin{itemize}

 \item Sea $A\in M_{m\times n}(\mathbb{K})$,  $B\in M_{n\times p}(\mathbb{K})$ y $C\in M_{p\times q}(\mathbb{K})$.
 
 \item Se puede realizar el producto $AB$ , el resultado ser\'a una matriz $m\times p$ que se podr\'a multiplicar por $C$ y el producto $(AB)C$ ser\'a una matriz $m\times q$. 
 
\item  An\'alogamente, se puede realizar el producto $BC$ que dar\'a una matriz $n\times q$ y se puede realizar tambi\'en el producto $A(BC)$ que dar\'a una matriz $m\times q$.
 
 \item Entonces la propiedad se puede expresar como:
     \[(AB)C=A(BC) \]
\end{itemize}
 \end{frame} 
 
 
 
    \begin{frame}
  \frametitle{Ejemplos}
  
   \begin{block}{Ejercicios}
   Se consideran las matrices con coeficientes en $\mathbb{R}$:
  \[ A = \left(\begin{array}{cc}1 & 2 \\3 & 4\end{array}\right);
  B = \left(\begin{array}{ccc}1 & 0 & 1 \\0 & 1 & 0\end{array}\right);
  C = \left(\begin{array}{cc}1 & -1 \\0 & 1 \\-1 & 0\end{array}\right)
  \]
  Probar que:
      \[(AB)C=A(BC) \]

  \end{block}


 \end{frame} 
 
 
     \begin{frame}
  \frametitle{Ejemplos}
  
   \begin{block}{Soluci\'on}
  \[ (AB)C = \left(\begin{array}{ccc}1 & 2 &1 \\3 & 4 &3\end{array}\right) 
  	\left(\begin{array}{cc}1 & -1 \\0 & 1 \\-1 & 0\end{array}\right) =
	\left(\begin{array}{cc}0 & 1 \\0 & 1 \end{array}\right)
  \]

  \[ A(BC) =\left(\begin{array}{cc}1 & 2 \\3 & 4\end{array}\right) 
  	\left(\begin{array}{cc}0 & -1 \\0 & -1 \end{array}\right) =
	\left(\begin{array}{cc}0 & 1 \\0 & 1 \end{array}\right)
  \]
  \end{block}


 \end{frame} 
 
 
 %%Exemple 2
 
   \begin{frame}
  \frametitle{Ejemplos}
  
   \begin{block}{Ejemplo}
   \[AI_n = A \]
   \[ I_nB = B \]
  \end{block}
   \begin{itemize}

 \item Sea $A\in M_{m\times n}(\mathbb{K})$ y $B\in M_{n\times p}(\mathbb{K})$.
 
 \item Se puede realizar el producto $AI_n$ y el resultado ser\'a una matriz $m\times n$. 
 
\item  An\'alogamente, se puede realizar el producto $I_nB$ y el resultado ser\'a una matriz $n\times p$.
 
 \item Adem\'as se puede comprobar que se verifica que:
   \[AI_n = A\ \mathrm{i}\ I_nB = B \]
\end{itemize}
 \end{frame} 
 
 
 
    \begin{frame}
  \frametitle{Ejemplos}
  
   \begin{block}{Ejercicios}
   Consid\'erense las matrices con coeficientes en $\mathbb{R}$:
  \[
  A = \left(\begin{array}{ccc}1 & 0 & 1 \\0 & 1 & 0\end{array}\right);
  B = \left(\begin{array}{cc}1 & -1 \\0 & 1 \\-1 & 0\end{array}\right)
  \]
  Probar que:
   \[AI_3 = A \]\[ I_3B = B \]

  \end{block}
 \end{frame} 


     \begin{frame}
  \frametitle{Ejemplos}
  
   \begin{block}{Soluci\'on}
  \[ AI_3= \left(\begin{array}{ccc}1 & 0 & 1 \\0 & 1 & 0\end{array}\right) 
  	\left(\begin{array}{ccc}1 & 0&0 \\0 & 1 &0 \\0 & 0 &1\end{array}\right) =
	A
  \]

  \[ I_3B =\left(\begin{array}{ccc}1 & 0&0 \\0 & 1 &0 \\0 & 0 &1\end{array}\right)
  \left(\begin{array}{cc}1 & -1 \\0 & 1 \\-1 & 0\end{array}\right) = B
  \]
  \end{block}


 \end{frame} 
 
 
      \begin{frame}
  \frametitle{Ejemplos}
  
   \begin{block}{Nota importante}
N\'otese que, en particular, para matrices cuadradas $A\in M_n(\mathbb{K})$, $I_n$ es un elemento neutro del producto, es decir:
\[AI_n = I_nA = A\]
para toda matriz cuadrada $A$ de orden $n$.
  \end{block}


 \end{frame} 
 
 
 %excepcions
 
       \begin{frame}
  \frametitle{Excepciones}
  En general, \alert<1>{no se cumplen} las siguientes propiedades:
  
   \begin{block}{Propiedad conmutativa}
La multiplicaci\'on de matrices no es conmutativa. 
  \end{block}
\[A = \left(\begin{array}{cc}0 & 1 \\0 & 0\end{array}\right), B = \left(\begin{array}{cc}0 & 0 \\1 & 0\end{array}\right) \in M_2(\mathbb{R}) \]
\[AB = \left(\begin{array}{cc}1 & 0 \\0 & 0 \end{array}\right) \neq \left(\begin{array}{cc}0 & 0 \\0 & 1\end{array}\right) = BA\]
 \end{frame} 
 
        \begin{frame}
  \frametitle{Excepciones}
  En general, \alert<1>{no se cumplen} las siguientes propiedades:
  
   \begin{block}{Ley de simplificaci\'on}
No se cumple la ley de simplificaci\'on en un producto. 
 \end{block}
\[A = \left(\begin{array}{cc}0 & 1 \\0 & 2\end{array}\right), B = \left(\begin{array}{cc}1 & 1 \\3 & 4\end{array}\right), C = \left(\begin{array}{cc}2 & 5 \\3 & 4\end{array}\right)  \in M_2(\mathbb{R}) \]
satisfacen:
\[AB = AC\]
pero en cambio $B\neq C$.
 \end{frame} 
 
 
         \begin{frame}
  \frametitle{Excepciones}
  En general, \alert<1>{no se cumplen} las siguientes propiedades:
  
   \begin{block}{Divisores de cero}
Existen divisores de cero, es decir:
\[AB=0 \notimplies A = 0 \ o \ B = 0.\]
\end{block}
Por ejemplo:
\[A = \left(\begin{array}{cc}0 & 3\\0 & 0\end{array}\right)\neq 0, B = \left(\begin{array}{cc}0 & 2 \\0 & 0\end{array}\right), A,B  \in M_2(\mathbb{R}) \]
pero en cambio:
\[AB = \left(\begin{array}{cc}0 & 0\\0 & 0\end{array}\right) \]

 \end{frame} 
 
 %%proposici�
          \begin{frame}
  \frametitle{Matrices diagonales}
  
   \begin{block}{Proposici\'on}
Sean $A,B$ dos matrices cuadradas de orden $n$.
\begin{itemize}
\item Si $A$ y $B$ son matrices diagonales con diagonales $a_{11},a_{22},\cdots,a_{nn}$ i $b_{11},b_{22},\cdots,b_{nn}$ respectivamente, entonces $A$ y $B$ conmutan y la matriz producto $AB = BA$ tambi\'en es diagonal con diagonal $a_{11}b_{11}, a_{22}b_{22},\cdots, a_{nn}b_{nn}$.
\item Si $A$ y $B$ son matrices triangulares superiores (inferiores) entonces el producto $AB$ es tambi\'en una matriz triangular superior (inferior).
\end{itemize}
\end{block}
 \end{frame} 
 
%transposades
\begin{frame}
  \frametitle{Matrices transpuestas}
  
   \begin{block}{Transpuesta de una matriz}
Sea $A = (a_{ij})_{m\times n}\in M_{m\times n} (\mathbb{K})$. Se denomina transpuesta de la matriz $A$ y se denota como $A^t$ a la matriz $A^t = (a_{ji})_{n\times m}\in M_{n\times n}(\mathbb{K})$. 

Es decir, la matrix obtenida a partir de $A$ intercambiando filas por columnas. 
\end{block}

Por ejemplo, la transpuesta de la matriz $A =  \left(\begin{array}{ccc}1&0 & 3\\ 2&1 & -1\end{array}\right)$ es $A^t =  \left(\begin{array}{cc}1 & 2\\ 0 & 1\\3 &-1\end{array}\right)$
 \end{frame} 
 
 \begin{frame}
  \frametitle{Matrices transpuestas}
  Entre las propiedades de las matrices transpuestas destacan las siguientes:
   \begin{block}{Idempotencia}
Para toda matriz $A$:
\[(A^t)^t = A\]
\end{block}
 \end{frame} 


 \begin{frame}
  \frametitle{Matrices transpuestas}
   Entre las propiedades de las matrices transpuestas destacan las siguientes:
   \begin{block}{Transpuesta de una suma}
Si $A$ y $B$ son matrices del mismo orden $m\times n$, entonces $(A+B)^t = A^t+B^t$. 
Es decir, la transpuesta de una suma de matrices es la matriz obtenida por la suma de sus respectivas transpuestas. Adem\'as, el resultado se puede generalizar a $r$ sumandos y se tiene que si $A_i$ son todas del mismo orden, entonces:
\[\displaystyle(\sum_{i=1}^r A_i)^t = \sum_{i=1}^r A_i^t\]
\end{block}
 \end{frame} 



 \begin{frame}
  \frametitle{Matrices transpuestas}
  Entre las propiedades de las matrices transpuestas destacan las siguientes:
   \begin{block}{Transpuesta de una suma}
Si $A\in M_{m\times n}(\mathbb{K})$ y $B\in M_{n\times p}(\mathbb{K})$, entonces la transpuesta del producto de $A$ por $B$ es el producto de las transpuestas pero con el orden cambiado, es decir:
\[(AB)^t = B^tA^t\in M_{p\times m}(\mathbb{K})\]
\end{block}
 \end{frame} 
 
 %quadrades
 
 \begin{frame}
  \frametitle{Matrices cuadradas}
Para acabar esta secci\'on volvemos a las matrices cuadradas con unas cuantas definiciones adicionales. Primero n\'otese que la transposici\'on, en el caso de matrices cuadradas, es una operaci\'on interna, es decir:

  \begin{block}{Transposici\'on como operaci\'on interna}
La transpuesta de una matriz cuadrada $A\in M_n(\mathbb{K})$  es otra matriz cuadrada $A^t\in M(\mathbb{K})$.
\end{block}

 Entonces tienen sentido las siguientes definiciones:
 \end{frame} 


 \begin{frame}
  \frametitle{Matrices cuadradas}
Sea $A=(a_{ij})\in M_n(\mathbb{K})$ una matriz cuadrada. D\'icese que:
  \begin{block}{Matriz sim\'etrica}
$A$ es sim\'etrica si coincide con su transpuesta, esto causa la simetr\'ia de la matriz respecto a su diagonal.
\[A \ \mathrm{simetrica} \Longleftrightarrow A = A^t \Longleftrightarrow a_{ij} = a_{ji}\ \forall\ i,j \]
\end{block}
 \end{frame} 
 
  \begin{frame}
  \frametitle{Matrices cuadradas}
Sea $A=(a_{ij})\in M_n(\mathbb{K})$ una matriz cuadrada. D\'icese que:
 
  \begin{block}{Matriz antisim\'etrica}
$A$ es antisim\'etrica si su transpuesta coincide con su opuesta, lo cual exige que la diagonal est\'e compuesta \'unicamente por ceros y que los elementos sim\'etricos sean opuestos entre s\'i.
\[A \ \mathrm{antisimetrica} \Longleftrightarrow A^t = -A \Longleftrightarrow a_{ij} = -a_{ji}\ \forall\ i,j \]
\end{block}
 \end{frame} 



 \begin{frame}
  \frametitle{Matrices cuadradas}
Sea $A=(a_{ij})\in M_n(\mathbb{K})$ una matriz cuadrada. D\'icese que:
  \begin{block}{Matriz regular}
$A$ es invertible o regular si existe otra matriz cuadrada $A^{-1} \in M_n(\mathbb{K})$ tal que $AA^{-1} = A^{-1}A = I_n$. Cuando existe esta matriz $A^{-1}$ es siempre \'unica, con la propiedad mencionada y se llama matriz inversa de $A$.
\end{block}

N\'otese que no basta con cumplir solo $AA^{-1} = I_n$ (o solo $A^{-1}A = I_n$) ya que el producto no es en general conmutativo. Por tanto, la matriz inversa ha de verificar que los resultados de los dos productos son la matriz identidad.

 \end{frame} 
 
 

 \begin{frame}
  \frametitle{Matrices cuadradas}
Sea $A=(a_{ij})\in M_n(\mathbb{K})$ una matriz cuadrada. D\'icese que:
 
  \begin{block}{Matriz singular}
$A$ es singular si no tiene inversa, es decir, cuando no es regular.
\end{block}


  \begin{block}{Matriz ortogonal}
$A$ es ortogonal si es regular y adem\'as su inversa coincide con su transpuesta. Dicho de otra manera:
\[A \ \mathrm{ortogonal} \Longleftrightarrow AA^t = A^tA= I_n \]

\end{block}
 \end{frame} 



 \begin{frame}
  \frametitle{Matrices cuadradas}
Se verifica el siguiente resultado respecto a las matrices inversas:
  
  \begin{block}{Proposici\'on}
Sea $A,B\in M_n(\mathbb{K})$. Entonces si $A$ y $B$ son invertibles, tambi\'en lo es su producto: 
\[(AB)^{-1} = B^{-1}A^{-1}\]
\end{block}

 \end{frame} 
 
 \begin{frame}
  \frametitle{En resumen}
  
  Las operaciones anteriores conforman el llamado \'algebra matricial. Este nombre es adecuado ya que gracias a ellas es posible realizar la manipulaci\'on habitual de ecuaciones con matrices igual que se hace con n\'umeros reales siempre y cuando se tenga precauci\'on con aquellas propiedades que no se verifican, vistas ellas anteriormente.
  
  Por ejemplo, en una ecuaci\'on con matrices todo lo que est\'e sumando pasa al otro t\'ermino restando y viceversa. 
  
  De esta manera se pueden resolver ecuaciones del tipo: encuentre una matriz $X$ tal que $A+2X =3B$ donde $A$ y $B$ son matrices conocidas. La soluci\'on ser\'a $X=1/2(3B-A)$.
  \end{frame}
  
   \begin{frame}
  \frametitle{En resumen}
   N\'otese sin embargo que las ecuaciones de la forma $AX=B$ no se pueden manipular de la forma habitual a no ser que la matriz $A$ sea cuadrada e invertible. Entonces se tendr\'a $X=A^{-1}B$. N\'otese que valdr\'ia multiplicar a la izquierda por $A^{-1}$ pero no valdr\'ia hacerlo a la derecha. Si la ecuaci\'on que se tiene es de la forma $XA=B$ entonces, si $A$ es invertible, ser\'a $X=BA^{-1}$, multiplicando a la derecha por $A^{-1}$.
\end{frame}

\begin{frame}
\frametitle{En resumen}
   
   Se puede calcular tambi\'en las potencias n-\'esimas de las matrices de la forma habitual $A^n = A\cdot A\cdot \cdots \cdot A$ ($n$ veces). N\'otese que el binomio de Newton para calcular $(A+B)^n$ solo se verifica en los casos en que $A$ y $B$ conmuten. Por ejemplo:
   \[(A+B)^2 = A^2+AB+BA+B^2\] 
Si $A$ y $B$ conmutan, entonces $(A+B)^2 = A^2+2AB+B^2$.

Pasa exactamente lo mismo con las potencias sucesivas.
 \end{frame} 
 
\section{Operaciones elementales}
\subsection{Matrices escalonadas}

 \begin{frame}
  \frametitle{Matrices escalonadas}
Vamos a introducir ahora un tipo especial de matrices triangulares superiores (inferiores), las llamadas matrices escalonadas por filas (por columnas).

 \begin{block}{Matriz escalonada}
Una matriz $A\in M_{m\times n}(\mathbb{K})$ es escalonada por filas cuando se cumplen simult\'aneamente las dos condiciones siguientes:
\begin{itemize}
\item El primer elemento no nulo de cada fila, denominado pivote, est\'a a la derecha del pivote de la fila superior.
\item Las filas nulas est\'an en la parte inferior de la matriz.
\end{itemize}
\end{block}
\end{frame} 
 

 \begin{frame}
  \frametitle{Matrices escalonadas}

 \begin{block}{Ejemplo}
Estas matrices son escalonadas por filas:
\[ \left(\begin{array}{cccc}2 & 1 & 0 & 2 \\0 & 1 & 0 & 1 \\0 & 0 & 0 & 2 \\0 & 0 & 0 & 0\end{array}\right)
\left(\begin{array}{ccccc}3 & 4 & 2 & 1 & 5 \\0 & 2 & 1 & 0 & 1 \\0 & 0 & 3 & 1 & -1\end{array}\right)
\]
\end{block}
\end{frame} 


 \begin{frame}
  \frametitle{Matrices escalonadas}

 \begin{block}{Matriz escalonada reducida}
Una matriz $A\in M_{m\times n}(\mathbb{K})$ es escalonada  reducida por filas si es escalonada y adem\'as cumple los siguientes requisitos:
\begin{itemize}
\item Los pivotes son todos unos.
\item Todos los elementos que est\'an en la misma columna del pivote son nulos.
\end{itemize}
\end{block}
\end{frame} 



 \begin{frame}
  \frametitle{Matrices escalonadas}

 \begin{block}{Ejemplo}
Estas matrices son escalonadas reducidas por filas:
\[
\left(\begin{array}{cccc}1 & 0 & -1 & 0 \\0 & 1 & 0 & 0 \\0 & 0 & 0 & 0\end{array}\right)
\left(\begin{array}{cccc}1 & 0 & 0 & 4 \\0 & 1 & 0 & 0 \\0 & 0 & 1 & 1\end{array}\right)
\]
\end{block}
\end{frame} 



 \begin{frame}
  \frametitle{Matrices escalonadas}

 \begin{block}{Ejercicio}
 D\'e definiciones equivalentes para las matrices escalonadas por columnas y matrices escalonadas reducidas por columnas.

Ponga dos ejemplos de cada tipo de matriz.
 

\end{block}
\end{frame} 



 \begin{frame}
  \frametitle{Operaciones elementales de una matriz}

 \begin{block}{Operaciones elementales por filas}
Sea $A\in M_{m\times n}(\mathbb{K})$. Las siguientes operaciones se llaman operaciones elementales por filas de la matriz $A$:
\begin{itemize}
\item Multiplicar una fila por un $a\in\mathbb{K}, a\neq 0$.
\item Intercambiar dos filas.
\item Sumar un m\'ultiplo de una fila a otra.

\end{itemize}
\end{block}

De manera an\'aloga se pueden definir las operaciones elementales por columnas.
\end{frame} 
  
  
   \begin{frame}
  \frametitle{Matrices equivalentes}

 \begin{block}{Matrices equivalentes por filas}
Dos matrices $A,B\in M_{m\times n}(\mathbb{K})$ son equivalentes por filas (por columnas) si una de ellas se puede obtener a partir de la otra mediante un n\'umero finito de operaciones elementales por filas (columnas).
\end{block}


 \begin{block}{Teorema}
 \begin{itemize}
\item Toda matriz es equivalente por filas (columnas) a una matriz escalonada por filas (columnas).
\item Toda matriz es equivalente por filas (columnas) a una \'unica matriz escalonada reducida por filas (columnas).
\end{itemize}
\end{block}

\end{frame} 

 \begin{frame}
  \frametitle{Matrices equivalentes}
Lo demostraremos de manera constructiva. Es decir, hallaremos un algoritmo (m\'etodo de Gauss) para encontrar la matriz escalonada en cada caso.
 \begin{block}{Demostraci\'on}
 Sea $A = (a_{ij}) \in M_{m\times n}(\mathbb{K})$, entonces procederemos de la siguiente manera:
\begin{itemize}
\item[1] Si $a_{11}\neq 0$, se divide la primera fila por $a_{11}$ y se obtiene una matriz equivalente en la que $a_{11} = 1$. Entonces este nuevo $a_{11} = 1$ ser\'a el primer pivote. Ahora, se resta a cada fila $i$ la primera multiplicada por $a_{i1}$, la resta de elementos de la primera columna ser\'a 0 y se pasa al punto cuarto.
\end{itemize}
\end{block}
\end{frame} 


 \begin{frame}
  \frametitle{Matrices equivalentes}
 \begin{block}{Demostraci\'on}
\begin{itemize}
\item[2] Si $a_{11}= 0$, se busca el primer $i$ tal que $a_{i1}\neq 0$. Entonces se intercambian la primera fila y la $i$ obteniendo una matriz equivalente con un nuevo $a_{11} \neq 0$, volvemos al punto uno y repetimos el proceso.
\item[3] Si $a_{i1}= 0$,  para todo $i=1,\cdots,m$, entonces dejamos esta primera columna de ceros y aplicamos el algoritmo del paso uno a la matriz resultante de eliminira la primera columna.\item[4] Se repite el proceso a la matriz obtenida de eliminar la primera fila y la primera columna de nuestra matriz.
\end{itemize}
\end{block}
\end{frame} 
  
  
   \begin{frame}
  \frametitle{Matrices equivalentes}
N\'otese que con este m\'etodo se obtiene la \'unica matriz escalonada equivalente por filas cuyos pivotes son todos unos. Para obtener la matriz escalonada reducida, se aplica el primer algoritmo anterior con el fin de obtener la matriz escalonada por filas equivalente a la matriz dada. Enconces, si hay alg\'un elemento $a_{ij}$ distinto de cero en la columna de determinado pivote se resta a la fila de este elemento (la fila $i$) la fila del pivote multiplicada por $a_{ij}$ y con este m\'etodo se hacen cero todos los elementos situados por debajo de los pivotes.
\end{frame} 
  
  
  
    %%Exercicis
   \begin{frame}
  \frametitle{Ejercicios}
   \begin{block}{Ejercicio 1}
Consid\'erese la matriz $A\in M_{3\times 4}(\mathbb{R})$ dada por:
\[A= \left(\begin{array}{cccc}1 & -1 & -3 & 8 \\4 & -2 & -6 & 19 \\3 & -6 & -17 & 41\end{array}\right)\]
Calc\'ulese su matriz escalona y su escalonada reducida por filas. 
\end{block}
  \end{frame} 
  
     \begin{frame}
  \frametitle{Ejercicios}

En este caso, $a_{11}=1\neq0$ y por lo tanto podemos poner ceros mediante la resta con la primera columna:
\[A \sim f_2 = f_2-4f_1\]
\[A\sim f_3 = f_3-3f_1\]
  \end{frame}
  
  
       \begin{frame}
  \frametitle{Ejercicios}

\[\left(\begin{array}{cccc}1 & -1 & -3 & 8 \\0 & -2 & 6 & -13 \\0 & -3 & -8 & 17\end{array}\right)\]
Pasamos ahora al punto cuarto del algortimo; es decir, se aplica el mismo razonamiento a la matriz $2\times3$ obtenida por eliminaci\'on de la primera fila y de la primera columna.
  \end{frame}  
  
  
  \begin{frame}
  \frametitle{Ejercicios}
\[\left(\begin{array}{ccc} -2 & 6 & -13 \\ -3 & -8 & 17\end{array}\right)\]
En este caso $a_{11} = 2$, comenzamos por dividir la primera fila por 2, haciendo ceros con la resta de la primera columna:
\[A\sim f_1 = f_1/2\]
\[A\sim f_2 = f_2+3f_1\]
  \end{frame}  
  
    \begin{frame}
  \frametitle{Ejercicios}
De esta manera, una matriz escalonada por filas equivalente a $A$ ser\'ia:
\[A\sim \left(\begin{array}{cccc}1 & -1 & -3 & 8 \\0 & 1 & 3 & -13/2 \\0 & 0 & 1 & -5/2\end{array}\right)\]

  \end{frame}  
  
  
    \begin{frame}
  \frametitle{Ejercicios}
Para calcular la \'unica matriz escalonada reducida equivalente a $A$ solo hace falta hacer ceros por debajo de cada pivote:
\[A\sim f_2 = f_2-3f_3\]
\[A\sim f_1 = f_1+3f_3\]
\[A\sim f_1 = f_1+f_2\]
\[A\sim \left(\begin{array}{cccc}1 & 0 & 0 & 3/2 \\0 & 1 & 0 & 1 \\0 & 0 & 1 & -5/2\end{array}\right)\]
  \end{frame}  
    
       \begin{frame}
  \frametitle{Ejercicios}
   \begin{block}{Ejercicio 2}
Consid\'erese la matriz $A\in M_{3\times 5}(\mathbb{R})$ dada por:
\[A= \left(\begin{array}{ccccc}0 & 0 & -1 & -3 & 8 \\1 & 0 & -2 & -6 &1 \\3 & 0 & -6 & -15 & 4\end{array}\right)\]
Calc\'ulese su matriz escalonada y su escalonada reducida por filas. 
\end{block}
  \end{frame} 
    
    
    \begin{frame}
  \frametitle{Ejercicios}
En este caso $a_{11}=0$ pero en cambio $a_{21}\neq 0$, por tanto se comenzar\'a por intercambiar las filas 1 y 2, despu\'es se hacen ceros por debajo del primer pivote:
\[A\sim f_2 \rightarrow f_1\]
\[A\sim f_3 = f_3-3f_1\]
  \end{frame}     
    

    \begin{frame}
  \frametitle{Ejercicios}
\[A= \left(\begin{array}{ccccc}1 & 0 & -2 & -6 &1 \\ 0 & 0 & -1 & -3 & 8 \\ 0 & 0 & 0 & 3 & 1\end{array}\right)\]
Se llega al punto cuarto del algoritmo con la matriz siguiente:
\[A= \left(\begin{array}{cccc} 0 & -1 & -3 & 8 \\ 0 & 0 & 3 & 1\end{array}\right)\]
En esta matriz la primera columna est\'a formada completamente por ceros.
\end{frame}     
  
      \begin{frame}
  \frametitle{Ejercicios}
Seg\'un el algoritmo, lo que se ha de hacer es continuar con la matriz que se hace constar a continuaci\'on:
\[A= \left(\begin{array}{ccc}  -1 & -3 & 8 \\  0 & 3 & 1\end{array}\right)\]
En esta se divide la primera fila por $-1$:
\[A\sim f_1 /(-1)\]
\[A\sim f_2 /3\]

  \end{frame}     
        
 \begin{frame}
  \frametitle{Ejercicios}
Por tanto una matriz escalonada por filas equivalente a $A$ ser\'ia:
\[A\sim \left(\begin{array}{ccccc}1 & 0 & -2 & -6 &1 \\ 0 & 0 & 1 & 3 & -8 \\0 & 0 & 0 & 1 & 1/3\end{array}\right)\]
  \end{frame}     


  \begin{frame}
  \frametitle{Ejercicios}
Finalmente, para encontrar la matriz escalonada reducida por filas equivalente a $A$ se har\'a:
\[A\sim f_2 = f_2-3f_3\]
\[A\sim f_1 = f_1+6f_3\]
\[A\sim f_1 = f_1+2f_2\]

\[A\sim \left(\begin{array}{ccccc}1 & 0 & 0 & 0 &-15 \\ 0 & 0 & 1 & 0 & -9 \\0 & 0 & 0 & 1 & 1/3\end{array}\right)\]
  \end{frame}   
  
  \subsection{Rango de una matriz}
 
  \begin{frame}
  \frametitle{Rango de una matriz}
Dada la unicidad de la matriz escalonada reducida, se puede definir sobre una matriz $A$ mediante su matriz escalonada reducida por filas (por columnas) equivalente:
   \begin{block}{Rango}
Sea $A\in M_{m\times n}(\mathbb{K})$. Se denomina rango de $A$ y se denota como $rg(A)$, al número de filas no nulas que tiene su \'unica matriz escalonada reducida por filas equivalente.
\end{block}
  \end{frame}   

  \begin{frame}
  \frametitle{Rango de una matriz}
   \begin{block}{Teorema}
Sea $A\in M_{m\times n}(\mathbb{K})$. El rango de $A$ coincidir\'a con el n\'umero de columnas no nulas de su \'unica matriz escalonada reducida por columnas equivalente.
\end{block}
En realidad, el n\'umero de filas no nulas es siempre el mismo en cualquier matriz equivalente por filas (por columnas) a la dada. Por tanto, para calcular el rango de una matriz $A$ bastar\'a con encontrar una matriz $B$ escalonada por filas (columnas) equivalente a $A$ y contar el n\'umero de filas (columnas) no nulas de $B$.
  \end{frame}   
  
  
     \begin{frame}
  \frametitle{Ejercicio}
   \begin{block}{Ejericicio}
Calc\'ulese el rango de la matriz $A$ del ejercicio 1 anterior:
\[A= \left(\begin{array}{cccc}1 & -1 & -3 & 8 \\4 & -2 & -6 & 19 \\3 & -6 & -17 & 41\end{array}\right)\]
\end{block}
  \end{frame} 
  
       \begin{frame}
  \frametitle{Ejercicio}
Se ha visto en el ejercicio 1 que:
  \[A\sim \left(\begin{array}{cccc}1 & 0 & 0 & 3/2 \\0 & 1 & 0 & 1 \\0 & 0 & 1 & -5/2\end{array}\right)\]
  Y por lo tanto, $rg(A)$ es simplemente el n\'umero de filas no nulas, $rg(A)=3$
  \end{frame} 
  
     \begin{frame}
  \frametitle{Ejercicios}
   \begin{block}{Ejercicio 4}
Calc\'ulese el rango de la matriz $B$
\[B= \left(\begin{array}{ccc}1 & 2 & 1 \\0 & 2 & 3 \\1 & 4 & 4\end{array}\right)\]
\end{block}
  \end{frame} 


     \begin{frame}
  \frametitle{Ejercicios}
Se hacen las siguientes operaciones:
\[B\sim f_3 = f_3-f_1\]
\[B\sim f_2 = f_2/2\]
\[B\sim f_3 = f_3-2f_2\]
\[B\sim \left(\begin{array}{ccc}1 & 2 & 1 \\0 & 1 & 3/2 \\0 & 0 & 0\end{array}\right)\]
Por tanto la matriz escalonada obtenida tiene dos filas no nulas y consecuentemente $rg(B)=2$.
  \end{frame} 
  
  
  
  \subsection{C\'alculo de la matriz inversa}
       \begin{frame}
  \frametitle{Caracterizaci\'on de las matrices invertibles}
Con las matrices escalonadas y las operaciones elementales no solo se puede calcular el rango de una matriz sino que tambi\'en resultan \'utiles en el c\'alculo de matrices inversas como veremos a continuaci\'on. El primer aporte que pueden hacer es la caracterizaci\'on de las matrices invertibles a trav\'es de su rango y de su matriz escalonada reducida.
  \end{frame} 
  
    
     \begin{frame}
  \frametitle{Teorema de caracterizaci\'on}
   \begin{block}{Teorema}
Sea $A$ una matriz cuadrada $A\in M_n(\mathbb{K})$. Entonces son equivalentes:
\begin{itemize}
\item $A$ es invertible
\item $rg(A) = n$
\item La matriz escalonada reducida por filas (por columnas) equivalente a $A$ es la matriz identidad $I_n$
\end{itemize}
\end{block}
  \end{frame} 
  
       \begin{frame}
  \frametitle{Teorema de caracterizaci\'on}
Adem\'as, la tercera equivalencia aporta un m\'etodo para calcular la matriz inversa de una matriz invertible $A \in M_n(\mathbb{K})$. Este consiste en escribir la matriz identidad $I_n$ a la derecha de la matriz (escrito de forma abreviada $(A | I_n)$) y, a trav\'es de transformaciones elementales por filas (o por columnas), calcular la matriz escalonada reducida que ser\'a de la forma$(I_n | B)$. La matriz $B$ resultante es precisamente la matriz inversa de $A$, es decir $A^{-1} = B$.
    \end{frame} 


     \begin{frame}
  \frametitle{C\'alculo de la matriz inversa}
   \begin{block}{Ejercicio 5}
Sea $A$ la matriz cuadrada $A\in M_n(\mathbb{K})$ dada por:
\[ \left(\begin{array}{ccc}1 & 2 & 1 \\0 & 2 & 3 \\1 & -1 & 2\end{array}\right)\]
Raz\'onese si $A$ es invertible y, si lo es, calc\'ulese su inversa.
\end{block}
  \end{frame} 
  
    \begin{frame}
  \frametitle{C\'alculo de la matriz inversa}
Se comenzar\'a por calcular su rango mediante las operaciones elementales siguientes:
\[A\sim f_3 = f_3-f_1\]
\[A\sim f_2 = f_2/2\]
\[A\sim f_3 = f_3+3f_2\]
\[A\sim f_3 = 2f_3/11\]
\[A\sim \left(\begin{array}{ccc}1 & 2 & 1 \\0 & 1 & 3/2 \\0 & 0 & 1\end{array}\right)\]
Se obtiene una matriz escalonada con tres filas no nulas. Por lo tanto $rg(A)=3$ y la matriz $A$ es invertible.
  \end{frame} 

  
  
      \begin{frame}
  \frametitle{C\'alculo de la matriz inversa}
Para calcular la matriz inversa se har\'a:
\[ \left(\begin{array}{ccccccc}1 & 2 & 1 & | & 1&0&0 \\0 & 2 & 3 &|&0&1&0 \\1 & -1 & 2&|&0&0&1\end{array}\right)\]

\[A\sim f_3 = f_3-f_1\]
\[A\sim f_2 = f_2/2\]
\[A\sim f_3 = f_3+3f_2\]
\[A\sim f_3 = 2f_3/11\]
\[ \left(\begin{array}{ccccccc}1 & 2 & 1 & | &1&0&0 \\0 & 1 & 3/2 &|&0&1/2&0 \\0 & 0 & 1&|&-2/11&3/11&2/11\end{array}\right)\]
  \end{frame} 


     \begin{frame}
  \frametitle{C\'alculo de la matriz inversa}

\[A\sim f_2 = f_2-3f_3/2\]
\[A\sim f_1 = f_1-f_3\]
\[A\sim f_1 = f_1-2f_2\]
\[ \left(\begin{array}{ccccccc}1 & 0 & 0 & | &7/11&-5/11&4/11 \\0 & 1 & 0 &|&3/11&1/11&-3/11 \\0 & 0 & 1&|&-2/11&3/11&2/11\end{array}\right)\]
  \end{frame} 
  
       \begin{frame}
  \frametitle{C\'alculo de la matriz inversa}
Por tanto:
\[A^{-1}= \left(\begin{array}{ccc}7/11&-5/11&4/11 \\3/11&1/11&-3/11 \\-2/11&3/11&2/11\end{array}\right) =
\frac{1}{11}\left(\begin{array}{ccc}7&-5&4 \\3&1&-3 \\-2&3&2\end{array}\right) \]
  \end{frame} 
  
    \section{Ecuaciones y sistemas lineales}
  \subsection{Ecuaciones matriciales}

       \begin{frame}
  \frametitle{?`Qu\'e es una ecuaci\'on matricial?}
Una equaci\'on matricial es una ecuaci\'on donde la inc\'ognita es una matriz. Se resuelven transformando la ecuaci\'on inicial en otra equivalente utilizando las propiedades y definiciones vistas. Para hallar la inc\'ognita es necesaria la matriz inversa. 
  \end{frame} 
  
       \begin{frame}
    \frametitle{?`Qu\'e es una ecuaci\'on matricial?}
       \begin{block}{M\'etodo de resoluci\'on}
\begin{itemize}
\item[]        \[XP = Q-R\]
\item  Se multiplica por la derecha en ambos t\'erminos por $P^{-1}$
\[XPP^{-1} = (Q-R)P^{-1}\]
\item Por definici\'on de matriz inversa $AA^{-1} = A^{-1}A = I_n$
 \[XI_n = (Q-R)P^{-1}\]
\item Por propiedad de la matriz identidad $AI_n = I_nA = A$
\[X=(Q-R)P^{-1}\]  
\end{itemize}
\end{block}
  \end{frame} 
  
  
  
         \begin{frame}
    \frametitle{?`Qu\'e es una ecuaci\'on matricial?}
       \begin{block}{Ejercicios}
Resu\'elvase la ecuaci\'on matricial $P+QX=RS-TX$. ?`Qu\'e condici\'on ha de cumplirse para que se pueda hallar $X$?
\end{block}
  \end{frame} 



  
         \begin{frame}
    \frametitle{?`Qu\'e es una ecuaci\'on matricial?}
\[ P+QX=RS-TX \]\[ (P-P)+QX+TX=RS-P+(-TX+TX)\]
\[ QX+TX=RS-P \]\[ (Q+T)X=RS-P\]
\[  (Q+T)^{-1}(Q+T)X=(Q+T)^{-1}(RS-P)\]\[  I_nX=(Q+T)^{-1}(RS-P) \]
\[  X=(Q+T)^{-1}(RS-P) \]
Para poder hallar $X$ es necesario que la matriz $(Q+T)$ tenga inversa. 
  \end{frame} 


  \subsection{Sistemas de ecuaciones lineales}
         \begin{frame}
    \frametitle{Sistemas de ecuaciones lineales}
       \begin{block}{Definici\'on}
Un sistema de $m$ ecuaciones lineales con $n$ inc\'ognites de la forma:
\[\left\{\begin{array}{ccccccccc}
a_{11}x_1 & + & a_{12}x_2 & + & \cdots & + & a_{1n}x_n & = & b_1\\
a_{21}x_1 & + & a_{22}x_2 & + & \cdots & + & a_{2n}x_n & = & b_2\\
\cdots &  & \cdots &  & \cdots &  & \cdots &  & \cdots\\
a_{m1}x_1 & + & a_{m2}x_2 & + & \cdots & + & a_{mn}x_n & = & b_m\\
\end{array}\right.\]
Con $a_{ij}, b_i\in \mathbb{K}$ por $i=1,2,\cdots,m$ i $j=1,2,\cdots,n$ es conocido con el nombre de \textbf{sistema de ecuacions lineales}. Una soluci\'on de este sistema es un conjunto de $n$ valores $\alpha_i\in \mathbb{K}, i=1,2,\cdots,n$ tales que al hacer las sustituciones $x_i=\alpha_i$ en cada una de las $m$ ecuaciones las convierten en identidades.
\end{block}
  \end{frame} 
  
  
  
           \begin{frame}
    \frametitle{Sistemas de ecuaciones lineales}
       \begin{block}{Forma matricial}
Un sistema de ecuaciones se puede escribir en forma matricial como $AX=B$ donde: 
\[A=\left(\begin{array}{cccc}
a_{11} & a_{12}  & \cdots  & a_{1n} \\
a_{21} & a_{22}  & \cdots  & a_{2n} \\
\vdots &  \vdots & \ddots  & \vdots \\
a_{m1} & a_{m2} & \cdots &  a_{mn}\\
\end{array}\right)
X=\left(\begin{array}{c}
x_1 \\x_2\\\vdots \\x_m
\end{array}\right)
B=\left(\begin{array}{c}
b_1 \\b_2\\\vdots \\b_m
\end{array}\right)
\]
La matriz $A$ se denomina \textbf{matriz de coeficientes}, la matriz $B$ \textbf{de t\'erminos independientes} y la matriz $X$ \textbf{de inc\'ognitas}.

\end{block}
  \end{frame} 
  
  
             \begin{frame}
    \frametitle{Sistemas de ecuaciones lineales}
       \begin{block}{Matriz ampliada}
Dado el sistema matricial $AX=B$, se define la \textbf{matriz ampliada} del sistema como: 
\[(A|B)\]
como ya se ha visto en la secci\'on anterior.
\end{block}
  \end{frame} 
  
    
             \begin{frame}
    \frametitle{Sistema de ecuaciones lineales}
       \begin{block}{Nota}
Si resulta que $m=n$, entonces el sistema de ecuaciones lineales se puede resolver f\'acilmente:
\[AX=B\]
\[A^{-1}AX=A^{-1}B\]
\[X=A^{-1}B\]
Basta encontrar la matriz inversa, si existe, y multiplicar las dos matrices $A^{-1}$ y $B$.
\end{block}
  \end{frame} 
  
  
               \begin{frame}
    \frametitle{Sistemas de ecuaciones lineales}
       \begin{block}{Sistemas compatibles e incompatibles}
Un sistema de $m$ ecuaciones lineales y $n$ inc\'ognitas $AX=B$ es:
\begin{itemize}
\item \textbf{Compatible}: si tiene al menos una soluci\'on.
\begin{itemize}
\item \textbf{Determinado}: si la soluci\'on es \'unica.
\item \textbf{Indeterminat}: si tiene infinitas soluciones.
\end{itemize}\item \textbf{Incompatible}: si no tiene soluci\'on.
\end{itemize}
\end{block}
Dos sistemes lineales del mismo tama\~no (es decir, los dos tienen el mismo n\'umero de ecuaciones y el mismo n\'umero de inc\'ognitas) son \textbf{equivalentes} si tienen las mismas soluciones.

  \end{frame} 
  
  
  
    
               \begin{frame}
    \frametitle{Sistemas de ecuaciones lineales}
       \begin{block}{Ejercicios}
Resu\'elvase el sitema lineal siguiente:
\[\left\{\begin{array}{ccccccc}
x_1 & + & x_2 & + & 2x_3 & = & 9\\
2x_1 & + & 4x_2 & - & 3x_3 & = & 1\\
3x_1 & + & 6x_2 & - &  5x_3 & = & 0\\
\end{array}\right.\]
\end{block}
  \end{frame} 



    
               \begin{frame}
    \frametitle{Sistemas de ecuaciones lineales}
Se obtendr\'an una serie de sistemas lineales equivalentes m\'as sencillos.
\begin{itemize}
\item La primera ecuaci\'on queda igual y se eliminar\'a $x_1$ de las otras dos ecuaciones.
\item A la segunda ecuaci\'on se le suma la primera multiplicada por $-2$.
\item A la tercera ecuaci\'on se le suma la primera multiplicada por $-3$.
\end{itemize}

\[\left\{\begin{array}{ccccccc}
x_1 & + & x_2 & + & 2x_3 & = & 9\\
 &  & 2x_2 & - & 7x_3 & = & -17\\
 &  & 3x_2 & - &  11x_3 & = & -27\\
\end{array}\right.\]

  \end{frame} 

               \begin{frame}
    \frametitle{Sistemas de ecuaciones lineales}
    De forma an\'aloga se eliminar\'a la variable $x_2$ de la tercera ecuaci\'on, sum\'andole la segunda multiplicada por $-3/2$
    
\[\left\{\begin{array}{ccccccc}
x_1 & + & x_2 & + & 2x_3 & = & 9\\
 &  & 2x_2 & - & 7x_3 & = & -17\\
 &  & & - &  \frac{1}{2}x_3 & = & -\frac{3}{2}\\
\end{array}\right.\]

  \end{frame} 
  
                 \begin{frame}
    \frametitle{Sistemas de ecuaciones lineales}
El sistema se puede resolver c\'omodamente por la sustituci\'on regresiva:
    \begin{itemize}
\item Se obtiene $x_3$ de forma directa de la tercera equaci\'on.
\[x_3 = 3\]
\item Se sustituye $x_3$ en la segunda ecuaci\'on y se halla $x_2$.
\[x_2 = \frac{-17+7\cdot 3}{2} = 2\]
\item Se sustituye $x_3$ y $x_2$ en la primera ecuaci\'on y se encuentra $x_1$.
\[x_1 = 9-2-2\cdot3 = 1\]
\end{itemize}
  \end{frame} 


  \subsection{El m\'etodo de Gauss}

               \begin{frame}
    \frametitle{El m\'etodo de de Gauss}
    Ya se han visto las matrices escalonadas en una secci\'on anterior. 
       \begin{block}{Definici\'on}
Un sistema $HX=C$ es escalonado si la matriz ampliada $(H|C)$ es una matriz escalonada.

Las variables que se corresponen con los pivotes se llaman \textbf{variables dependientes} y el resto \textbf{variables independientes}.
\end{block}

Ya se ha visto el m\'etodo de Gauss para obtener matrices escalonadas y para calcular el rango de una matriz $A$. Ahora se aplicar\'a para encontrara las coluciones de los sistemas de ecuaciones lineales.
  \end{frame} 
  
               \begin{frame}
    \frametitle{El m\'etodo de Gauss}
       \begin{block}{M\'etodo de Gauss}
\begin{enumerate}
\item Entre todas las filas se seleccionar\'a la que tenga el pivote los m\'as a la izquierda posible y se colocar\'a como primera fila.
\item Con el pivote de la primera fila se reducir\'an a cero todos los elementos que se encuentren por debajo de \'el.
\item Se repiten los pasos uno y dos con la submatriz formada por todas las filas excepto la primera. La nueva matriz que se obtiene tendr\'a ceros por debajo del pivote de la fila 2.
\item Se repite el proceso con el fin de obtener una matriz escalonada.
\end{enumerate}
\end{block}
  \end{frame} 
  

\begin{frame}
\frametitle{El m\'etodo de Gauss}
\begin{block}{Ejercicios}
Resu\'elvase el siguiente sistema:
\[
\left\{\begin{array}{ccccccc}x_1 & + & x_2 &   &   & = & 3 \\  &   & x_2 & + & x_3 & = & 5 \\x_1 &   &   & + & x_3 & = & 4 \\5x_1 & - & x_2 & + & x_3 & = & 6\end{array}\right.
\]

\end{block}
  \end{frame} 
  
                 \begin{frame}
    \frametitle{El m\'etodo de Gauss}
Se realizan las siguientes operaciones:
\[f_3 \sim f_3-f_1; f_4 \sim f_4-5f_1\]
\[f_3 \sim f_3+f_2; f_4 \sim f_4+6f_1\]
\[f_4 \sim f_4-\frac{7}{2}f_3\]
Se obtiene as\'i una matriz escalonada con tres variables dependientes $x_1,x_2$ y $x_3$ ya que tiene tres pivotes.

\[
\left(\begin{array}{ccccc}1 & 1 & 0 & | & 3 \\ 0   & 1 & 1 & | & 5 \\
0 &0  & 2& | & 6 \\
0 &0 & 0  & | & 0\end{array}\right)
\]
  \end{frame} 

  
                   \begin{frame}
    \frametitle{El m\'etodo de Gauss}
Aplicando sustituci\'on regresiva se obtiene esto: 
\[x_1 = 1, x_2 = 2, x_3 = 3\]
  \end{frame} 
  
  
  
  
                 \begin{frame}
    \frametitle{El m\'etodo de Gauss}
       \begin{block}{Ejercicios}
Resu\'elvase el siguiente sistema:
\[
\left\{\begin{array}{ccccccccc}x_1 & - & 2x_2 & +  & x_3  & - &x_4& = & 3 \\
2x_1  & -  & 3x_2 & + & 2x_3 & -&x_4 & = & -1 \\
3x_1 &  - &  5x_2 & + & 3x_3 & - &4x_4& = & 3 \\
-x_1 & + & x_2 & - & x_3 & +&2x_4 & = & 5\end{array}\right.
\]

\end{block}
  \end{frame} 
  
  
  \begin{frame}
    \frametitle{El m\'etodo de Gauss}
Se hace lo siguiente:
\[f_2 \sim f_2-2f_1; f_3 \sim f_3-3f_1; f_4 \sim f_4+f_1\]
\[f_3 \sim f_3-f_2; f_4 \sim f_4+f_2\]
Se obtiene una matriz escalonada con dos variables dependientes $x_1$ y $x_2$ y dos variables independientes $x_3$ y $x_4$.

\[
\left(\begin{array}{cccccc}1 &-2 & 1 & -1&| & 4 \\ 0   & 1 & 0& -1 & | & -9 \\
0 &0 &0  & 0& | & 0 \\
0 &0 & 0&0  & | & 0\end{array}\right)
\]
  \end{frame} 
  
                     \begin{frame}
    \frametitle{El m\'etodo de Gauss}
Aplicando sustituci\'on regresiva se obtienen las variables dependientes en funci\'on de las independientes.
\[x_1 = -14-x_3+3x_4\]
\[ x_2 = -9+x_4\]
\[ x_3 = x_3\]
\[ x_4 = x_4\]
  \end{frame} 
  
  
  
  \section{Determinantes}
  \subsection{El concepto de determinante}
  
       \begin{frame}
  \frametitle{Concepto de determinante}
   \begin{block}{Definici\'on}
Dada una matriz cuadrada $A\in M_n(\mathbb{K})$ con $a_{ij}\in \mathbb{K}$ se denomina determinante de la matriz $A$ y se denota como $|A|$ o como $det(A)$ a un elemento del cuerpo $\mathbb{K}$ que se define por inducci\'on de la forma siguiente:
\begin{itemize}
\item Si $n=1$, $A=(a_{11})$ entonces $|A| = a_{11}$.
\item Si $n>1$, $|A| = a_{11}\alpha_{11}-a_{12}\alpha_{12}+\cdots+(-1)^{n+1}a_{1n}\alpha_{1n}$ 
\end{itemize} 
Donde $\alpha_{1i}$ es el determinante de la matriz de orden n-1 que se obtiene al suprimir las primera fila y la columna $i$-\'esima de la matriz $A$.
\end{block}
  \end{frame} 
  
  
       \begin{frame}
  \frametitle{Concepto de determinante}
   \begin{block}{Ejercicio 1}
C\'alculese el determinante de una matriz cuadrada de orden 2 gen\'erica:
\[A=  \left(\begin{array}{cc}a_{11}&a_{12} \\a_{21}&a_{22} \end{array}\right) \]
\end{block}
Se tiene que:
\[\alpha_{11} = |a_{22}| = a_{22}\]
\[\alpha_{12} = |a_{21}| = a_{21}\]
Y por lo tanto el determinante es:
\[|A| = a_{11}\alpha_{11}-a_{12}\alpha_{12} = a_{11}a_{22}-a_{12}a_{22}\]
  \end{frame} 
  
         \begin{frame}
  \frametitle{Concepto de determinante}
   \begin{block}{Ejercicio 2}
Calc\'ulese el determinante de la matriz cuadrada de orden 3 gen\'erica:
\[A=  \left(\begin{array}{ccc}a_{11}&a_{12}&a_{13} \\a_{21}&a_{22}&a_{23} \\a_{31}&a_{32}&a_{33} \end{array}\right) \]
\end{block}
Se tiene que:
\[\alpha_{11} = \left|\begin{array}{cc}a_{22} & a_{23}\\a_{32} & a_{33} \end{array}\right|, 
\alpha_{12} = \left|\begin{array}{cc}a_{21} & a_{23}\\a_{31} & a_{33} \end{array}\right|,
\alpha_{13} = \left|\begin{array}{cc}a_{21} & a_{22}\\a_{31} & a_{32} \end{array}\right|\]
Y por tanto el determinante es:
\[|A| = a_{11}(a_{22}a_{33}-a_{23}a_{32})-a_{12}(a_{21}a_{33}-a_{31}a_{23})+a_{13}(a_{21}a_{32}-a_{31}a_{22})\]
Este es el mismo resultado que se obtiene con la regla de Sarrus.
  \end{frame} 
  
           \begin{frame}
  \frametitle{Concepto de determinante}
   \begin{block}{Ejercicio 3}
Calc\'ulese el determinante de la transpuesta de la matriz cuadrada de orden 2 gen\'erica:
\[A^t=  \left(\begin{array}{cc}a_{11}&a_{21} \\a_{12}&a_{22} \end{array}\right) \]
\end{block}
Se tiene que:
\[\alpha_{11} =|a_{22}|, \alpha_{12} =|a_{12}|\]
Y por tanto el determinante es:
\[|A^t| = a_{11}\alpha_{11}-a_{21}\alpha_{12} = a_{11}a_{22}-a_{21}a_{12}\]

Por tanto se concluye que $|A^t| = |A|$.

Este resultado es cierto $\forall n\geq 1$: si $A\in M_n(\mathbb{K})$, $|A^t| = |A|$
  \end{frame} 
  
    \subsection{Propiedades}
           \begin{frame}
  \frametitle{Propiedades de los determinantes}
 Debido a que $A\in M_n(\mathbb{K})$, $|A^t| = |A|$, se puede deducir toda una serie de propiedades tanto por filas como por columnas.
 
 Se denota como $det(u_1,\cdots, u_i,\cdots, u_n)$ al determinante de la matriz $A\in M_n(\mathbb{K})$ que tiene como filas (o columnas) las matrices fila (o columna) $u_i$, $i=1,\cdots, n$.

    \end{frame} 
  
  
             \begin{frame}
  \frametitle{Propiedades de los determinantes}
   \begin{block}{Propiedad 1}
   \[det(u_1, \cdots,\lambda u_i, \cdots , u_n) = \lambda det(u_1,\cdots, u_i, \cdots, u_n) \]
   En particular si $\lambda=0$, entonces $det(u_1,\cdots,0,\cdots,u_n) = 0$.
\end{block}

\[ \left|\begin{array}{ccc}5 & 2 & 1 \\10 & 4 & 4 \\1 & 2 & 1\end{array}\right|
=  2\left|\begin{array}{ccc}5 & 1 & 1 \\10 & 2 & 4 \\1 & 1 & 1\end{array}\right|
=  4\left|\begin{array}{ccc}5 & 1 & 1 \\5 & 1 & 2 \\1 & 1 & 1\end{array}\right|
\]
  \end{frame} 
  
  
  
             \begin{frame}
  \frametitle{Propiedades de los determinantes}
   \begin{block}{Propiedad 2}
   \[det(u_1, \cdots,u_i+u_i', \cdots , u_n) =\] \[ det(u_1,\cdots, u_i, \cdots, u_n) + det(u_1,\cdots, u_i', \cdots, u_n) \]
\end{block}

\[ \left|\begin{array}{ccc}2 & 1+2 & 1 \\3 & 3+2 & 1 \\1 & 1+2 & 2\end{array}\right|
= \left|\begin{array}{ccc}2 & 1 & 1 \\3 & 3 & 1 \\1 & 1 & 2\end{array}\right|
+ \left|\begin{array}{ccc}2 & 2 & 1 \\3 & 2 & 1 \\1 & 2 & 2\end{array}\right|
\]
  \end{frame} 
  
  
 \begin{frame}
  \frametitle{Propiedades de los determinantes}
   \begin{block}{Propiedad 3}
   Si se intercambian dos filas o columnas de un determinante, el determinante cambia de signo.
   \[det(u_1, \cdots,u_i, \cdots, u_j, \cdots , u_n) = - det(u_1, \cdots,u_j, \cdots, u_i, \cdots , u_n) \]
\end{block}

\[ \left|\begin{array}{ccc}1 & 2 & 1 \\0 & 1 & 1 \\2 & 3 & 0\end{array}\right|
= - \left|\begin{array}{ccc}1 & 2 & 1 \\2 & 3 & 0\\0 & 1 & 1 \end{array}\right|
= \left|\begin{array}{ccc}1 & 2 & 1 \\0 & 3 & 2\\1 & 1 & 0 \end{array}\right|
\]
  \end{frame} 
  
  
   \begin{frame}
  \frametitle{Propiedades de los determinantes}
   \begin{block}{Propiedad 4}
   Si una matriz tiene dos filas o columnas iguales, su determinante es nulo.
   \[det(u_1, \cdots,u_i, \cdots, u_i, \cdots , u_n) = 0 \]
\end{block}

  \begin{block}{Propiedad 5}
   Si una matriz tiene dos filas o columnas proporcionales, su determinante es nulo.
   \[det(u_1, \cdots,u_i, \cdots, \lambda u_i, \cdots , u_n) = 0 \]
\end{block}
  \end{frame} 
  
  
     \begin{frame}
  \frametitle{Propiedades de los determinantes}
   \begin{block}{Propiedad 6}
   Si una fila o columna es combinaci\'on lineal de las otras, el determinante es nulo. Es decir, si $u_i = \sum_{k\neq i} a_k u_k$ entonces:
   \[det(u_1, \cdots,u_i, \cdots, u_n) = 0 \]
   
\end{block}

\[ \left|\begin{array}{ccc}1 & 0 & 1 \\2 & 1 & 3 \\3 & 4 & 7\end{array}\right| = 0\]
En este caso la tercera columna es la suma de las dos anteriores.
   \end{frame} 
  
  
    
     \begin{frame}
  \frametitle{Propiedades de los determinantes}
   \begin{block}{Propiedad 7}
   
El determinante no cambia si a una fila o columna se le suma una combinaci\'on lineal de las otras.
   \[det(u_1, \cdots,u_i, \cdots, u_n) = det(u_1, \cdots,u_i + \sum_{k\neq i} a_k u_k, \cdots, u_n) \]
   
\end{block}

\[ \left|\begin{array}{ccc}3 & 1 & -1 \\5 & 0 & 2 \\1 & -1 & 1\end{array}\right| =  
	\left|\begin{array}{ccc}3 & 1 & -1 \\5 & 0 & 2 \\9 & 0 & 2\end{array}\right| \]
Dado que hemos obtenido el segundo determinante a partir del primero sumando a la tercera fila una suma de la primera y la segunda.
   \end{frame} 
  
        \subsection{Matriz adjunta}
 \begin{frame}
  \frametitle{Adjuntos de una matriz}
   \begin{block}{Definici\'on}
   Sea $A=(a_{ij})_{n\times n}, n\geq 2$. Sea $a_{ii}$ el elemento que ocupa la fila $i$ y la columna $j$ de la matriz $A$. Si se suprime la fila $i$ y la columna $j$ de $A$ se obtiene una matriz cuadrada de orden $n-1$.
   \begin{itemize}
   \item El determinante de esta matriz, que se denotar\'a como $\alpha_{ij}$ y se llama menor complementario de $a_{ij}$.
    \item El elemento $A_{ij} = (-1)^{i+j}\alpha_{ij}$ se denomina adjunto de $a_{ij}$.
    \item La matriz adjunta de $A=(a_{ij})_{n\times n}, n\geq 2$, es la matriz que tiene como coeficientes a los adjuntos $A_{ij}$ de los elementos $a_{ij}$ de la matriz $A$. Se denota por $adj(A)$.
   \end{itemize}
   \end{block}
   \end{frame} 
   
   
    \begin{frame}
  \frametitle{Adjuntos de una matriz}
   \begin{block}{Ejercicio 4}
Calc\'ulese la matriz adjunta de $A$
\[A = \left(\begin{array}{ccc}1 & 0 & 3 \\2 & 1 & 1 \\3 & -1 & 2\end{array}\right)\]
   \end{block}
   \end{frame}
   
       \begin{frame}
  \frametitle{Adjuntos de una matriz}
Soluci\'on
\[adj(A) = \left(\begin{array}{ccc}3 & -1 & -5 \\-3 & -7 & 1 \\-3 & 5 & 1\end{array}\right)\]
   \end{frame}  
  
      \subsection{C\'alculo de un determinante}

       \begin{frame}
  \frametitle{C\'alculo de un determinante}
El determinante de una matriz cuadrada $A = (a_{ij} )$ de tipo $n \times n$ con $n \geq 2$ se puede calcular desarrollando por los adjuntos de los elementos de cualquiera de sus filas o columnas.
   \end{frame}  
  

    \begin{frame}
  \frametitle{C\'alculo de un determinante}
   \begin{block}{Desarrollar un determinante por adjuntos}
Sea  $A = (a_{ij} )$ una matriz cuadrada de orden $n \times n$. Entonces se verifica:
\[det(A) = a_{i1}A_{i1} + a_{i2}A_{i2} + \cdots + a_{in}A_{in}\]
(desarrollo de un determinante por los adjuntos de los elementos de una fila) y tambi\'en:
\[det(A)=a_{1j}A_{1j} +a_{2j}A_{2j} +\cdots+a_{nj}A_{nj}\]
(desarrollo por los adjuntos de los elementos de una columna).
   \end{block}
   \end{frame}

\begin{frame}
  \frametitle{C\'alculo de un determinante}
     \begin{block}{Ejercicio 5}
Calc\'ulese el determinante siguiente:
\[|A| = \left|\begin{array}{ccc}1 & 0 & 3 \\2 & 1 & 1 \\3 & -1 & 2\end{array}\right|\]
Desarr\'ollese por los elementos de la primera fila.
\end{block}
\end{frame} 

\begin{frame}
  \frametitle{C\'alculo de un determinante}
\[|A| = 1\left|\begin{array}{cc} 1 & 1 \\ -1 & 2\end{array}\right|
-0\left|\begin{array}{cc} 2 & 1 \\ 3 & 2\end{array}\right|
+3\left|\begin{array}{cc} 2 & 1 \\ 3 & -1\end{array}\right| = 3-0-15 = -12\]
\end{frame}   


\begin{frame}
  \frametitle{C\'alculo de un determinante}
     \begin{block}{Ejercicio 6}
Calc\'ulese el determinate siguiente:
\[|A| = \left|\begin{array}{ccc}1 & 0 & 3 \\2 & 1 & 1 \\3 & -1 & 2\end{array}\right|\]
Desarr\'ollese por los elementos de la segunda columna.
\end{block}
\end{frame} 

\begin{frame}
  \frametitle{C\'alculo de un determinante}
\[|A| = -0\left|\begin{array}{cc} 2 & 1 \\ 3 & 2\end{array}\right|
+1\left|\begin{array}{cc} 1 & 3 \\ 3 & 2\end{array}\right|
-(-1)\left|\begin{array}{cc} 1 & 3 \\ 2 & 1\end{array}\right| = 0-7-5 = -12\]
\end{frame}   



\begin{frame}
  \frametitle{C\'alculo de un determinante}
     \begin{block}{Ejemplo}
     Si se aplican estos desarrollos a las matrices triangulares se tiene que el determinante de una matriz triangular es igual al producto de los elementos de la diagonal principal.

\end{block}

\[|A| = \left|\begin{array}{cccc}4 & 3 & 1 & 0 \\0 & -1 & 2 & 5 \\0 & 0 & -3 & 3  \\0 & 0 & 0 & 1\end{array}\right| = 
 1\left|\begin{array}{cccc}4 & 3 & 1  \\0 & -1 & 2 \\0 & 0 & -3 \end{array}\right|\]
 \[
 =  -3\left|\begin{array}{cccc}4 & 3   \\0 & -1  \end{array}\right|
 =  -3(-4) = 12
 \]
En cada paso se ha hecho el desarrollo del determinante por los adjuntos de la \'ultima fila.
\end{frame} 



\begin{frame}
  \frametitle{C\'alculo de un determinante}
El desarrollo de un determinante por los adjuntos de los elementos de una fila o de una columna, junto con las propiedades mencionadas arriba, permiten simplificar de manera considerable el c\'alculo de un determinante.
\end{frame} 


\begin{frame}
  \frametitle{C\'alculo de un determinante}
     \begin{block}{Ejercicio 7}
Calc\'ulese el determinante:
\[|A| = \left|\begin{array}{ccc}3 & 1 & -1 \\5 & 0 & 2 \\1 & -1 & 1\end{array}\right|\]
Us\'ense las propiedades de los determinantes para ello.
\end{block}
\end{frame} 

\begin{frame}
  \frametitle{C\'alculo de un determinante}
\[|A| = \left|\begin{array}{ccc}3 & 1 & -1 \\5 & 0 & 2 \\1 & -1 & 1\end{array}\right| = 
\left|\begin{array}{ccc}3 & 1 & -1 \\5 & 0 & 2 \\4 & 0 & 0\end{array}\right| = 
-1 \left|\begin{array}{cc}5 & 2 \\4 & 0\end{array}\right| = 8 \]

Primero se suma la primera fila a la tercera y despu\'es se ha desarrollado por los adjuntos de los elementos de la segunda columna.
\end{frame} 

\begin{frame}
  \frametitle{C\'alculo de un determinante}
     \begin{block}{Ejercicio 8}
Calc\'ulese el determinante que figura a continuaci\'on:
\[|A| = \left|\begin{array}{ccc}a+b+c & b+c+a & c+b+a \\a & b & c \\1 & 1 & 1\end{array}\right|\]
Us\'ense las propiedades de los determinantes para ello.
\end{block}
\end{frame} 


\begin{frame}
  \frametitle{C\'alculo de un determinante}
\[|A| = \left|\begin{array}{ccc}a+b+c & b+c+a & c+b+a \\a & b & c \\1 & 1 & 1\end{array}\right| = \]
\[
(a+b+c) \left|\begin{array}{ccc}1 & 1 & 1 \\a & b & c \\1 & 1 & 1\end{array}\right|  = 0\]
Se ha extra\'ido factor com\'un $a + b + c$ y el determinante resultante es nulo porque tiene dos filas iguales.
\end{frame} 

\begin{frame}
  \frametitle{C\'alculo de un determinante}
     \begin{block}{Ejercicio 9}
Calc\'ulese el determinante
\[|A| = \left|\begin{array}{cccccc}1 & 2 & 1 &1&1&1 \\ 0 & 3 & 0 &0&0&0 \\ -1 & 1 & 0 &1&0&-1 \\
1 & 0 & 0 &0&0&1 \\ 0 & 0 &1 &0&0&0 \\  0 & 0 & 0 &0&0&1 \\ \end{array}\right|\]
Us\'ense las propiedades de los determinantes para ello.
\end{block}
\end{frame} 

\begin{frame}
  \frametitle{C\'alculo de un determinante}
\[|A| = \left|\begin{array}{cccccc}1 & 2 & 1 &1&1&1 \\ 0 & 3 & 0 &0&0&0 \\ -1 & 1 & 0 &1&0&-1 \\1 & 0 & 0 &0&0&1 \\ 0 & 0 &1 &0&0&0 \\  0 & 0 & 0 &0&0&1 \\ \end{array}\right| = 1\left|\begin{array}{ccccc}1 & 2 & 1 &1&1 \\ 0 & 3 & 0 &0&0 \\ -1 & 1 & 0 &1&0 \\1 & 0 & 0 &0&0 \\ 0 & 0 &1 &0&0 \end{array}\right| =\]\[ 1\left|\begin{array}{ccccc} 0 & 3 & 0 &0 \\ -1 & 1 & 0 &1 \\1 & 0 & 0 &0 \\ 0 & 0 &1 &0 \end{array}\right| = 1\left|\begin{array}{ccccc}  3 & 0 &0 \\  1 & 0 &1 \\ 0 &1 &0 \end{array}\right| = 3\left|\begin{array}{ccccc}   0 &1 \\ 1 &0 \end{array}\right| = -3 \]
Se ha desarrollado por la sexta fila, por la quinta columna, por la tercera fila y por la primera fila respectivamente.
\end{frame} 


\begin{frame}
  \frametitle{C\'alculo de un determinante}
     \begin{block}{Exercici 10}
Calc\'ulese el determinante de Vandermonde de orden 4. En el primer paso a cada fila se le resta la anterior multiplicada por $a$.
\[|A| = \left|\begin{array}{cccc}1 & 1 & 1 & 1 \\ a & b & c & d \\ a^2 & b^2 & c^2 & d^2 \\a^3 & b^3 & c^3 & d^3 \end{array}\right|\]
\end{block}
\end{frame} 

\begin{frame}
  \frametitle{C\'alculo de un determinante}
\[|A| = \left|\begin{array}{cccc}1 & 1 & 1 & 1 \\ a & b & c & d \\ a^2 & b^2 & c^2 & d^2 \\a^3 & b^3 & c^3 & d^3 \end{array}\right| = \left|\begin{array}{cccc}1 & 1 & 1 & 1 \\ 0 & b-a & c-a & d-a \\ 0 & b^2-ba & c^2-ca & d^2-da \\0 & b^3-b^2a & c^3-c^2a & d^3-d^2a \end{array}\right| = \]
\[1\left|\begin{array}{ccc} b-a & c-a & d-a \\ b^2-ba & c^2-ca & d^2-da \\b^3-b^2a & c^3-c^2a & d^3-d^2a \end{array}\right| = \]
\[\left|\begin{array}{ccc} b-a & c-a & d-a \\ b(b-a) & c(c-a) & d(d-a) \\b^2(b-a) & c^2(c-a) & d^2(d-a) \end{array}\right| =\]

\end{frame} 


\begin{frame}
  \frametitle{C\'alculo de un determinante}
\[ (b-a)(c-a)(d-a)\left|\begin{array}{ccc} 1 & 1 & 1 \\ b & c & d \\b^2 & c^2 & d^2 \end{array}\right| =\]
\[ (b-a)(c-a)(d-a)\left|\begin{array}{ccc} 1 & 1 & 1 \\ 0 & c-b & d-b \\0 & c^2- cb & d^2-db \end{array}\right| =\]
\[ (b-a)(c-a)(d-a)\left|\begin{array}{cc}  (c-b) & (d-b) \\ c(c- b) & d(d-b) \end{array}\right| =\]
\[ (b-a)(c-a)(d-a)(c-b)(d-b)\left|\begin{array}{cc}  1 & 1 \\ c & d \end{array}\right| =\]
\[ (b-a)(c-a)(d-a)(c-b)(d-b) (d-c)\]

\end{frame} 


\subsection{Aplicaciones de los determinantes}

\begin{frame}
  \frametitle{Aplicaciones en el c\'alculo de matrices}
Los determinantes son \'utiles para encontrar la inversa de una matriz, el rango de una matriz y, como veremos m\'as adelante, para la resoluci\'on de sistemas lineales.
\end{frame}   
  
  \begin{frame}
  \frametitle{Aplicaciones en el c\'alculo de matrices}
     \begin{block}{Teorema}
Sean $A$ y $B$ matrices cuadradas de orden $n$, $A,B\in M_n(\mathbb{K})$. Entonces:
\begin{itemize}
\item $A$ es invertible si y solo si $|A|\neq 0$
\item $|AB| = |A| \cdot |B|$
\item Si $|A|\neq0$, entonces $|A^{-1}| = \frac{1}{|A|}$
\item Si $|A|\neq0$, entonces $|A^{-1}| = \frac{(adj A)^t}{|A|}$
\end{itemize}
\end{block}

N\'otese que la \'ultima propiedad supone una nueva manera de calcular la matriz inversa de una matriz dada $A$. Para calcular la matriz inversa se ha de calcular la matriz adjunta, transponerla y dividirla por el determinante de la matriz dada.
\end{frame} 
  
  \begin{frame}
  \frametitle{Aplicaciones en el c\'alculo de matrices}
     \begin{block}{Ejercicio 10}
Calc\'ulese la matriz inversa de la matriz $A$ con esta nueva t\'ecnica:
\[A=\left(\begin{array}{ccc}1 & 0 & 2 \\-5 & 1 & -1 \\2 & -1 & 2\end{array}\right)\]
\end{block}
\end{frame}   
  
  
  
    \begin{frame}
  \frametitle{Aplicaciones en el c\'alculo de matrices}
    La matriz es invertible ya que $|A| = 7\neq 0$. 
\[(adjA)^t=\left(\begin{array}{ccc}1 & -2 & -2 \\8 & -2 & -9 \\3 & 1 & 1\end{array}\right)\]
\[A^{-1}=\frac{1}{7}\left(\begin{array}{ccc}1 & -2 & -2 \\8 & -2 & -9 \\3 & 1 & 1\end{array}\right)\]
Adem\'as se puede afirmar que $det(A^{-1}) = 7$.
\end{frame}   
  
    \begin{frame}
  \frametitle{Aplicaciones en el c\'alculo de matrices}
  Otra aplicaci\'on de los determinantes se halla en el c\'alculo del rango de una matriz.
     \begin{block}{Menores de orden $k$}
Sea $A$ una matriz de orden $m\times n$, $A=(a_{ij})\in M_{m\times n}(\mathbb{K})$ y sea $k<n$
\begin{itemize}
\item Se denomina menor de orden $k$ de la matriz $A$ al determinante de cualquier matriz cuadrada de orden $k$ obtenida al suprimir $m-k$ filas y $n-k$ columnas de $A$. 
\item Dado un menor de orden $k$ de la matriz $A$, orlar este menor consiste en completarlo hasta lograr un menor de orden $k+1$ de $A$ con otra fila y otra columna de la matriz dada $A$.
\end{itemize}
\end{block}
\end{frame}   
  
    
    \begin{frame}
  \frametitle{Aplicaciones en el c\'alculo de matrices}
Se pueden utilizar estos menores para calcular el rango de una matriz $A$ cualquiera.
     \begin{block}{Teorema}
Sea $A$ una matriz de orden $m\times n$, $A=(a_{ij})\in M_{m\times n}(\mathbb{K})$ y sea $k<n$
Entonces se puede encontrar un menor de orden $k$ no nulo y todos los de orden $k+1$ ser\'an nulos, entonces el $rang(A)=k$. Es decir, el rango de la matriz $A$ coincide con el orden del mayor menor no nulo obtenido de $A$.
\end{block}
\end{frame}  
  
     \begin{frame}
  \frametitle{Aplicaciones en el c\'alculo de matrices}
El teorema anterior se puede mejorar con el siguiente segmento:
     \begin{block}{Teorema}
Sea $A$ una matriz de orden $m\times n$, $A=(a_{ij})\in M_{m\times n}(\mathbb{K})$ y sea $k<n$
Entonces, se puede encontrar un menor de orden $k$ no nulo y todas las maneras posibles de orlar este menor dar\'an menores nulos, entonces el $rang(A)=k$.
\end{block}
\end{frame}   
  
  
     \begin{frame}
  \frametitle{Aplicaciones en el c\'alculo de matrices}
     \begin{block}{Ejercicio 12}
Calc\'ulese el rango de la siguiente matriz:
\[A = \left(\begin{array}{ccc}2 & -2 & -2 \\4 & -2 & -6 \\-1 & 1 & 1 \\0 & 1 & -1\end{array}\right)\]
\end{block}
\end{frame}  

    \begin{frame}
  \frametitle{Aplicaciones en el c\'alculo de matrices}
El primer menor de orden 2 es no nulo:
\[\left|\begin{array}{cc}2 & -2  \\4 & -2 \end{array}\right| = -2+8 = 6\neq0\]
Se sigue el primer teorema, hay que comprobar que todos los menosres de orden 3, si todos son 0 el rango ser\'a 2, y sino ser\'a tres (el m\'aximo ya que solo hay tres columnas).

Gracias al segundo teorema bastar\'a comprobar los menores de orden 3 que se obtienen orlando el menor no nulo hallado. 
\end{frame}  




   \begin{frame}
  \frametitle{Aplicaciones en el c\'alculo de matrices}
De esta manera solo hay que probar los menores:
\[\left|\begin{array}{ccc}2 & -2 & -2 \\4 & -2 & -6 \\-1 & 1 & 1\end{array}\right|=\cdots=0\]
\[\left|\begin{array}{ccc}2 & -2 & -2 \\4 & -2 & -6 \\0 & 1 & -1\end{array}\right|=\cdots=0\]
De esta forma se puede afirmar que $rang(A) = 2$.
\end{frame}  


     \begin{frame}
  \frametitle{Aplicaciones en el c\'alculo de matrices}
     \begin{block}{Ejercicio 13}
Determ\'inese para qu\'e valores del par\'ametro $\alpha$ la siguiente matriz es invertible y, en los caoss en los que lo es, hallar la inversa
\[A = \left(\begin{array}{ccc} \alpha & 4 & 5 \\ -\alpha & 1 & 2 \\ -\alpha & -\alpha & 0 \end{array}\right)\]
\end{block}
\end{frame} 



     \begin{frame}
  \frametitle{Aplicaciones en el c\'alculo de matrices}
Para ser invertible el determinante ha de ser distinto de cero:
\[det(A) = \alpha(7\alpha-3)\]
As\'i la matriz es invertible si $\alpha \neq 0, 3/7$,  en este caso la inversa es: 
\[A^{-1} = \frac{1}{7\alpha+3} \left(\begin{array}{ccc} 2 & -5 & 3/\alpha \\ -2 & 5 & -7 \\ \alpha+1 & \alpha-4 & 5 \end{array}\right) \]
\end{frame} 


     \begin{frame}
  \frametitle{Aplicaciones en el c\'alculo de matrices}
     \begin{block}{Ejercicio 14}
Determ\'inese el rango de la siguiente matriz seg\'un los valores del par\'ametro $\alpha$
\[A = \left(\begin{array}{cccc} \alpha & 1 & \alpha+1 & 1 \\ 0& 2\alpha & \alpha-1 & 0 \\ 1&0 & 2\alpha & 1 \end{array}\right)\]
\end{block}
\end{frame} 


     \begin{frame}
  \frametitle{Aplicaciones en el c\'alculo de matrices}
Se ve de manera inmediata que el menor de orden 2 formado por las dos primeras columnas y las filas primera y tercera no son nulas, cualquiera que sea el valor de $\alpha$
\[\left|\begin{array}{cc} \alpha & 1 \\ 1&0 \end{array}\right| = -1 \neq 0\]
\end{frame} 


     \begin{frame}
  \frametitle{Aplicaciones en el c\'alculo de matrices}
Complet\'andolo con la cuarta columna, que es la m\'as sencilla por no tener par\'ametros, se obtiene:
\[\left|\begin{array}{ccc} \alpha & 1 &1\\ 0 & 2\alpha &0 \\ 1&0&1 \end{array}\right| = 2\alpha(\alpha-1)\]
De esta manera el rango de la matriz ser\'a 3 (no puede ser superior porque la matriz solo tiene 3 filas) siempre que $\alpha$ sea distinto de $0,1$.
\end{frame} 
  
  
     \begin{frame}
  \frametitle{Aplicaciones en el c\'alculo de matrices}
En el caso de que $\alpha = 0$, la matriz queda:
\[A = \left(\begin{array}{cccc} 0 & 1 & 1 & 1 \\ 0& 0 & -1 & 0 \\ 1&0 & 0 & 1 \end{array}\right)\]
Con menor no nulo:
\[ \left|\begin{array}{ccc} 0 & 1 & 1  \\ 0& 0 & -1  \\ 1&0 & 0  \end{array}\right| = 1\]
Por lo tanto en este caso, el rango de $A$ es tambi\'en 3.
\end{frame} 


     \begin{frame}
  \frametitle{Aplicaciones en el c\'alculo de matrices}
En el caso de que $\alpha = 1$, la matriz queda as\'i:
\[A = \left(\begin{array}{cccc} 1 & 1 & 2 & 1 \\ 0& 2 & 0 & 0 \\ 1&0 & 2 & 1 \end{array}\right)\]
En este caso los dos menores de orden 3 que comprenden nuestros menor de orden dos no nulo son 0. ya se sabe con la cuarta columna y con la tercera se tiene:
\[ \left|\begin{array}{ccc} 1 & 1 & 2  \\ 0& 2 & 0  \\ 1&0 & 2  \end{array}\right| = 0\]
Por tanto, en este caso, el rango de $A$  es 2.
\end{frame} 


     \begin{frame}
  \frametitle{Aplicaciones a la resoluci\'on de sistemas de ecuaciones lineales}
Ya hemos visto c\'omo resolver sistemas de ecuaciones lineales con el m\'etodo de Gauss. En esta secci\'on veremos como emplear los conocimientos que tenemos sobre determinantes para hacer esta tarea de otra manera. En particular nos ser\'an muy \'utiles para:
\begin{itemize}
\item Discutir un sistema; es decir, decidir si un sistema es o no compatible en funci\'on de los valores que tenga un par\'ametro dado.
\item Resolver el sistema en el caso de compatibilidad.
\end{itemize}
Para ello se introduce el Teorema de Rouch\'e-Frobenius.
\end{frame} 

     \begin{frame}
  \frametitle{Aplicaciones a la resoluci\'on de sistemas de ecuaciones lineales}
     \begin{block}{Rango de un sistema de ecuaciones lineal}
Sea $AX=B$ un sistema de ecuaciones lineales. Se denotar\'a por $rang(A)$ el rango del sistema y por $rang(A^*)$ el rango de la matriz ampliada $(A|B)$
\end{block}
Se satisface que: 
\[rang(A) \leq \min(n,m)\]
Donde $n$ es el n\'umero de ecuaciones y $m$ el n\'umero de inc\'ognitas.
\end{frame} 


     \begin{frame}
  \frametitle{Aplicaciones a la resoluci\'on de sistemas de ecuaciones lineales}
  Sea $AX=B$ un sistema de ecuaciones lineales con $m$ ecuaciones y $n$ inc\'ognitas. 

     \begin{block}{Teorema de Rouch\'e-Frobenius}

La condici\'on necesaria y suficiente para que el sistema sea compatible es que $rang(A) = rang(A^*)$.

Adem\'as, si coincide que $rang(A) = n$, el sistema es compatible determinado. En caso contrario, si $rang(A)<n$, el sistema es compatible indeterminado.


\end{block}
\begin{itemize}
\item $rang(A) = rang(A^*)$: Sistema compatible
\begin{itemize}
\item $rang(A) = rang(A^*) = n$: Sistema compatible determinado
\item $rang(A) = rang(A^*)<n$: Sistema compatible indeterminado
\end{itemize}
\item $rang(A) < rang(A^*)$: Sistema incompatible
\end{itemize}
\end{frame} 


     \begin{frame}
  \frametitle{Aplicaciones a la resoluci\'on de sistemas de ecuaciones lineales}
     \begin{block}{Ejercicios}
Disc\'utase el rango del sistema siguiente:
\[
\left\{\begin{array}{ccccccc}6x & - & y & + & 3z & = & 6 \\-6x & + & 8y &   &   & = & -10 \\2x & - & 5y & - & z & = & 4\end{array}\right.
\]
\end{block}
Sol: $rang(A)=2<3=rang(A^*)$: sistema incompatible.
\end{frame} 
  
  
  
       \begin{frame}
  \frametitle{Aplicaciones a la resoluci\'on de sistemas de ecuaciones lineales}
     \begin{block}{Ejercicios}
Dado el sistema: 
\[
\left\{\begin{array}{ccccccc}
a_{11}x &+ & a_{12}y & + & a_{13}z & = & b_1 \\
a_{21}x &+ & a_{22}y & + & a_{23}z & = & b_2 \\
a_{31}x &+ & a_{32}y & + & a_{33}z & = & b_3 \\
\end{array}\right.
\]
con 
\[\left|
\begin{array}{cc}
a_{11}&a_{12}\\a_{21}&a_{22}
\end{array}
\right|\neq0\]
y $rang(A^*) = 2$, ?`qu\'e se puede decir del sistema?
\end{block}
Sol: Sistema compatible indeterminado.
\end{frame} 
  
  
       \begin{frame}
  \frametitle{Aplicaciones a la resoluci\'on de sistemas de ecuaciones lineales}
  Un sistema homog\'eneo es aquel donde $B=0$. Al aplicar el teorema de Rouch\'e-Frobenius se observa que siempre tienen soluci\'on, ya que al estudiar la matriz ampliada no se a\~nade informaci\'on adicional a la matriz de coeficientes.
     \begin{block}{Ejercicios}
Disc\'utase el rango del sistema siguiente:
\[
\left\{\begin{array}{ccccccc}2x & + & y & - & z & = & 0 \\
x & + & 2y & +  & z  & = & 0 \\
3x & + & y & - & 2z & = & 0\end{array}\right.
\]
\end{block}
Sol: $rang(A)=2=rang(A^*)<n=3$: sistema compatible indeterminado.
\end{frame} 
  
         \begin{frame}
  \frametitle{Aplicaciones a la resoluci\'on de sistemas de ecuaciones lineales}
     \begin{block}{Ejercicios}
Disc\'utase el rango del sistema en funci\'on del par\'ametro $a$.
\[
\left\{\begin{array}{ccccccc}ax & + & y & + & z & = & 1 \\
x & + & 2y & +  & az  & = & 1 \\
2x & + & y & + & z & = & a\end{array}\right.
\]
\end{block}
Sol: $a\neq 1,2$: sistema compatible determinado.
$a=1$: sistema compatible indeterminado.
 $a=2$: sistema incompatible.
\end{frame} 


         \begin{frame}
  \frametitle{Regla de Cramer}
     \begin{block}{Sistemas de Cramer}
Un sistema $AX=B$ con $m$ ecuaciones y $n$ inc\'ognitas se denomina regular o de Cramer si $rang(A) = n = m$.
\end{block}
Dado que el n\'umero de ecuaciones e inc\'ognitas coinciden se trata de matrices cuadradas de tama\~no igual al rango de la matriz.


De forma trivial, cualquier sistema de Cramer es compatible y determinado. En este caso la regla de Cramer permite calcular la soluci\'on el sistema de forma bastante sencilla.

\end{frame} 



         \begin{frame}
  \frametitle{Regla de Cramer}
     \begin{block}{Regla de Cramer}
Un sistema regular $AX=B$ donde $A_i$ denota la columna i-\'esima de la matriz $A$, admite una soluci\'on \'unica dada por:
\[x_i = \frac{D_i}{D}, i = 1,2,\cdots,n\]
Donde: 
\[D=det(A_1,A_2,\cdots,A_n)\]
\[D_i=det(A_1,A_2,\cdots,A_{i-1},B,A_{i+1},\cdots,A_n)\]
\end{block}

\end{frame} 


         \begin{frame}
  \frametitle{Regla de Cramer}
     \begin{block}{Ejercicios}
Disc\'utase y resu\'elvase el siguiente sistema:

\[
\left\{\begin{array}{ccccccc}2x & + & 3y & - & z & = & 6 \\
x & - & 5y & +  & 2z  & = & -4 \\
3x & + & 2y & -& 3z & = & -6\end{array}\right.
\]
\end{block}
Sol: SCD con
\[x=1, y=3, z=5\]
\end{frame} 

         \begin{frame}
  \frametitle{Regla de Cramer}
     \begin{block}{Ejercicios}
Disc\'utase y resu\'elvase el siguiente sistema:

\[
\left\{\begin{array}{ccccccc}x & + & 2y & - & z & = & 10 \\
2x & - & 4y & -  & 2z  & = & 5 \\
x & + & y & +& z & = & 6\end{array}\right.
\]
\end{block}
Sol: SCD con
\[x=\frac{83}{16}, y=\frac{15}{8}, z=\frac{-17}{16}\]
\end{frame} 


         \begin{frame}
  \frametitle{Regla de Cramer}
     \begin{block}{Ejercicios}
Disc\'utase y resu\'elvase el siguiente sistema:

\[
\left\{\begin{array}{ccccccc}
x & + & 2y & + &2z & = & 2\\
3x & - & 2y & -  & z  & = & 5 \\
2x & - & 5y & +& 3z & = & -4\\
x &+ & 4y & +  & 6z  & = & 0 \\
\end{array}\right.
\]
\end{block}
Sol: SCD con
\[x=2, y=1, z=1\]
\end{frame} 





         \begin{frame}
  \frametitle{Regla de Cramer}
     \begin{block}{Ejercicios}
Disc\'utase y resu\'elvase el siguiente sistema:

\[
\left\{\begin{array}{ccccccc}
2x & + & y & - &z & = & 0\\
x & +& 2y & +  & z  & = & 0 \\
3x & + & y & -& 2z & = & 0
\end{array}\right.
\]
\end{block}
Sol: SCI con
\[x=z, y=-z, z=z\]
\end{frame} 






         \begin{frame}
  \frametitle{Regla de Cramer}
     \begin{block}{Ejercicios}
Disc\'utase y resu\'elvase el siguiente sistema:

\[
\left\{\begin{array}{ccccccccccc}
3x & - & 4y & + &3z & -&s&+&2t &= & 0\\
3x & - & 6y & + &5z & -&2s&+&4t &= & 0\\
5x & - & 10y & + &7z & -&3s&+&t &= & 0\\
\end{array}\right.
\]
\end{block}
Sol: SCI con
\[x=x, y=\frac{x-5t}{2}, z=-5t, t=t, s=-3t\]
\end{frame} 




\end{document}